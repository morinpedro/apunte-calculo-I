\item Probar usando la definición que se cumplen las siguientes afirmaciones:

\begin{multicols}{2}
    \begin{enumerate}
        \item $\D \lim_{x\to 2} 3x-1=5 $.
        \item $\D \lim_{x\to 3} 4 x^2 = 36$.
        \item $\D \lim_{x\to 2} 3/x = 3/2$.
        \item $\D \lim_{x\to 1} x^2+2x = 3$.
    \end{enumerate}
\end{multicols}

\item Analizar la existencia de los siguientes límites:
\begin{multicols}{2}
    \begin{enumerate}
\item $\D \lim_{x\to 0} \sen \frac1x$;
\item $\D \lim_{x\to 0} \cos \frac1x$;
\item $\D \lim_{x\to 0} \sgn(x)$;
\item $\D \lim_{x\to 0} x \sen \frac1x$;
\item $\D \lim_{x\to 0} x^2 \sen \frac1x$;
\item $\D \lim_{x\to 0} (x + \sen \frac1x)$.
\end{enumerate}
\end{multicols}

\item ?`Para qué valores de $x_0\in\R$ existe $\limxo \lfloor x \rfloor$?

\item ?`Para qué valores de $x_0\in\R$ existe $\limxo \big(x - \lfloor x \rfloor\big)$?

\item Analizar la veracidad o falsedad de las siguientes afirmaciones:
\begin{enumerate}
    \item Si existe $\D\limxo \big(f(x)+g(x)\big)$, entonces existen
    $\D\limxo f(x)$ y $\D\limxo g(x)$.
    \item Si existe $\D\limxo \big(f(x)+g(x)\big)$ y existe
    $\D\limxo f(x)$, entonces existe $\D\limxo g(x)$.
    \item Si existe $\D\limxo \big(f(x)\cdot g(x)\big)$, entonces existen
    $\D\limxo f(x)$ y $\D\limxo g(x)$.
    \item Si existe $\D\limxo \big(f(x)\cdot g(x)\big)$ y existe
    $\D\limxo f(x)$, entonces existe $\D\limxo g(x)$.
    \item Si existe $\D\limxo \big(f(x)\cdot g(x)\big)$ y existe
    $\D\limxo f(x)\neq 0$, entonces existe $\D\limxo g(x)$.
    
\end{enumerate}

\item Sea $a>0$, $a\neq 1$ y supongamos que $\limxo f(x)=\ell>0$.
Demostrar que $\D\limxo \log_a(f(x)) = \log_a(\ell).$ (Recordar que $\D\log_a y = \frac{\ln y}{\ln a}$).

\item Consideremos la función constante $f(x)=a$, para algún $a\in\R$. Demostrar usando la definición que $\limxo f(x)=f(x_0)$, cualquiera sea $x_0\in\R$.
\item Consideremos la función \emph{identidad} $f(x)=x$. Demostrar usando la definición que $\limxo f(x)=f(x_0)$, cualquiera sea $x_0\in\R$.
\item Demostrar que si $a$ y $b$ son números reales, entonces la función lineal $f(x)=ax+b$ cumple que $\limxo f(x)=f(x_0)$, cualquiera sea $x_0\in\R$.
Usar lo demostrado en los ejercicios anteriores y las propiedades vistas en esta sección.
\item Demostrar que si $a$, $b$ y $c$ son números reales, entonces la función cuadrática $f(x)=ax^2+bx+c$ cumple que $\limxo f(x)=f(x_0)$, cualquiera sea $x_0\in\R$.
Usar lo demostrado en los ejercicios anteriores y las propiedades vistas en esta sección.
\item Demostrar que si $p(x)$ es una función polinomial, entonces $\limxo p(x)=p(x_0)$, cualquiera sea $x_0\in\R$.
\item Demostrar que si $p(x)$ y $q(x)$ son funciones polinomiales, entonces para la función racional $f(x)= \frac{p(x)}{q(x)}$ se cumple que $\limxo f(x)=f(x_0)$, cualquiera sea $x_0\in\R$, si $q(x_0)\neq0$.

