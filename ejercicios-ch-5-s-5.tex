\item * Si $f:\R\to\R$ es continua en $\R$ y derivable en todo punto de $\R\setminus\{a\}$, y existe $\D\lim_{x\to a}f'(x)= \ell$, entonces $f$ es derivable en $x=a$ y $f'(a) = \ell$.
Usar el Teorema del Valor Medio (de Lagrange) para calcular las derivadas laterales.

\item Probar las siguientes afirmaciones para una función $f:[a,b]\to \R$ continua en $[a,b]$ y derivable en $(a,b)$:
\begin{enumerate}
  \item Si $f'(x)\ge 0$ para todo $x\in(a,b)$, entonces $f$ es creciente en $[a,b]$; es decir, $a\le x_1<x_2\le b$ implica $f(x_1)\le f(x_2)$.
  \item Si $f$ es creciente, entonces $f'(x)\ge 0$ para todo $x\in(a,b)$.
  \item Si $f'(x)\le 0$ para todo $x\in(a,b)$, entonces $f$ es decreciente en $[a,b]$; es decir, $a\le x_1<x_2\le b$ implica $f(x_1)\ge f(x_2)$
  \item Si $f$ es decreciente, entonces $f'(x)\le 0$ para todo $x\in(a,b)$.
\end{enumerate}

\item Encontrar un ejemplo de una función $f:\R\to\R$ que sea estrictamente creciente y derivable, pero que no cumpla que $f'(x)>0$, para todo $x\in\R$.

\item Probar que la función $f(x)=\sqrt[3]{(x-3)^2}$ satisface $f(1)=f(5)$ pero no existe $c\in(1,5)$ tal que $f'(c)=0$. ?`Por qué no se puede aplicar el Teorema de Rolle?

\item Sea $f(x)=(x-1)\, x\, (x+2)\, (x+5)$. Probar que $f'(x)$ tiene tres raíces reales.

\item Para cada una de las siguientes funciones, determinar el punto $c\in(0,1)$ que cumple el Teorema del Valor Medio en el intervalo $[0,1]$:
\begin{multicols}{3}
  \begin{enumerate}
    \item $f(x)=3x$;
    \item $f(x)=2x^2$;
    \item $f(x)=\sqrt[3]{x}$.
  \end{enumerate}
\end{multicols}

\item Hallar $f'(x)$ para:
\begin{multicols}{2}

  \begin{enumerate}
    \item $\D f(x) = \arcsen(\sqrt x)$;
    \item $\D f(x) = \arccos(x^2+1)$;
    \item $\D f(x) = \big(e^{\arcsen (1/x)}\big)^x$;
    \item $\D f(x) = \sqrt[3]{\arccos\Big(\frac{x^2}{1+x^2}\Big)}$;
    \item $\D f(x) = \ln\big(\arccos (x)\big)$;
  \end{enumerate}
  
\end{multicols}