\usepackage{fullpage}
\usepackage[utf8]{inputenc}
\usepackage{amsmath,amsfonts,amsthm}
\usepackage[spanish]{babel}
\usepackage{enumerate}
\usepackage{graphicx,psfrag}
\usepackage{tikz}
% \usepackage{pgfplots}
\usepackage{stackengine}
\usepackage{xcolor}
\usepackage{comment}
\usepackage{array}
\usepackage{color}
\usepackage{srcltx}
\usepackage{todonotes}
\usepackage{multicol}
\usepackage{xspace}
\usepackage{stackengine}
\usepackage[%
%   pdftex=true,  % si se usa pdftex
   pdfpagemode=UseNone,
breaklinks=true,
%   pdfstartpage=3,   % para empezar en pag. 3
   bookmarks=true,
%   bookmarksopen=true,
%   bookmarksnumbered=true,
%   plainpages=false, % cuando existe p\'agina i y p\'agina 1
%   pdfstartview=XYZ,
%   pdfpagelabels=true, % ii en preambulo
%   pagebackref=true, % referencias a citas
   colorlinks,
   linkcolor=blue,% cambiar a black para imprimir
%   linktocpage, % link en el n\'umero de pagina en toc
   anchorcolor=blue,
   citecolor=black,% cambiar a black para imprimir
   filecolor=blue,
%    pagecolor=blue,
   urlcolor=blue
   ]{hyperref}

\usepackage{etoolbox}
\newcommand\getcurrentref[1]{%
 \ifnumequal{\value{#1}}{0}
  {??}
  {\the\value{#1}}%
}    

\usepackage{cancel}
%\newcounter{savectr}
%\newcommand{\savectr}[1]{\setcounter{savectr}{\value{#1}}}
%\newcommand{\backctr}[1]{\setcounter{#1}{\value{savectr}}}
\newcounter{contadorejemplo}
%\newcommand{\ejemplo}{\medskip\textbf{Ejemplo \arabic{contadorejemplo}:} \addtocounter{contadorejemplo}{1}}
%\setcounter{contadorejemplo}{1}
%\DeclareGraphicsExtensions{eps}

\newcommand{\comentario}[1]{{\begin{quote}\small\color{gray}#1\end{quote}}}


\newcommand{\completar}[1]{\todo[inline]{Completar. {#1}}}
\newcommand{\sucan}{\ensuremath{\left(a_n\right)_{n\in\N}}\xspace}
\newcommand{\sucxn}{\ensuremath{\left(x_n\right)_{n\in\N}}\xspace}
\newcommand{\sucyn}{\ensuremath{\left(y_n\right)_{n\in\N}}\xspace}
\newcommand{\sucSn}{\ensuremath{\left(S_n\right)_{n\in\N}}\xspace}
\newcommand{\serieak}{\ensuremath{\sum_{k=1}^\infty a_k}\xspace}
\newcommand{\seriean}{\ensuremath{\sum_{n=1}^\infty a_n}\xspace}
\newcommand{\seriebk}{\ensuremath{\sum_{k=1}^\infty b_k}\xspace}
\newcommand{\seriebn}{\ensuremath{\sum_{n=1}^\infty b_n}\xspace}
\newcommand{\subsucan}{\ensuremath{\left(a_{n_k}\right)_{k\in\N}}\xspace}
\newcommand{\sucbn}{\ensuremath{\left(b_n\right)_{n\in\N}}\xspace}
\newcommand{\succn}{\ensuremath{\left(c_n\right)_{n\in\N}}\xspace}
\newcommand{\niN}{{\ensuremath{n\in\N}}\xspace}
\newcommand{\xiR}{{\ensuremath{x\in\R}}\xspace}
\newcommand{\xoiR}{{\ensuremath{x_0\in\R}}\xspace}
\newcommand{\limxo}{{\ensuremath{\lim_{x\to x_0}}}}
\newcommand{\limho}{{\ensuremath{\lim_{h\to 0}}}}
\newcommand{\limxop}{{\ensuremath{\lim_{x\to x_0^+}}}}
\newcommand{\limxom}{{\ensuremath{\lim_{x\to x_0^-}}}}
\newcommand{\ton}{\stackunder{\,$\longrightarrow$\,}{\scriptsize$n\to\infty$}}
\newcommand{\tox}[1]{\stackunder{\,$\longrightarrow$\,}{\scriptsize$x\to{#1}$}}
\newcommand{\toy}[1]{\stackunder{\,$\longrightarrow$\,}{\scriptsize$y\to{#1}$}}
\newcommand{\eqLH}{\stackon{\,=}{\,\,\scriptsize{(L'H)}}\,\,}
\newcommand{\note}[1]%
{\noindent\centerline{\fbox{\parbox{.9\textwidth}{\textbf{#1}}}}}
\newcommand{\snote}[1]%
{\fbox{\textbf{#1}}}

% \theoremstyle{plain}
\parskip=10pt
\usepackage{thmtools}

\declaretheoremstyle[
  shaded={rulecolor=black, rulewidth=0.7pt, bgcolor=white, 
  textwidth=.95\textwidth, padding=6pt, leftmargin=5pt},
  spaceabove=6pt, spacebelow=6pt,
]{miestilo}
\declaretheorem[
  style=miestilo,
  name=Teorema,
  numberwithin=chapter,
  ]{theorem}
\declaretheorem[
    style=miestilo,
  name=Lema,
  numberlike=theorem,
  ]{lemma}
\declaretheorem[
    style=miestilo,
  name=Corolario,
  numberlike=theorem,
]{corollary}
\declaretheorem[
    style=miestilo,
    name=Proposición,
  numberlike=theorem,
  ]{proposition}
\declaretheorem[
    style=miestilo,
    name=Definición,
  numberlike=theorem,
  ]{definition}
%    \newtheorem{theorem}{Teorema}[chapter]
%    \newtheorem{lemma}[theorem]{Lema}
%    \newtheorem{corollary}[theorem]{Corolario}
%    \newtheorem{proposition}[theorem]{Proposición}

\theoremstyle{definition}
   \newtheorem{remark}[theorem]{Observación}
   \newtheorem{example}[theorem]{Ejemplo}
%    \newtheorem{definition}[theorem]{Definición}
%    \newtheorem{pb}{ }[chapter]
%    \newtheorem{pbm}[pb]{*}
%    \newtheorem{pbo}[pb]{**}

\usepackage{attachfile2}

%\usepackage{matlab-prettifier}
\usepackage{listings}
\lstset{
  basicstyle=\ttfamily,
  columns=fixed,
  fontadjust=true,
  basewidth=0.5em,
  frame=single,
  float=h!
}

%\lstset{language=Octave,basicstyle=\ttfamily}
\newcommand{\includescript}[2]{\lstinputlisting[title={Script \texttt{#2}. \attachfile{#1/#2}}]{#1/#2}}

\newcommand{\includefunction}[2]{\lstinputlisting[title={Función \texttt{#2}. \attachfile{#1/#2}}]{#1/#2}}


\newcommand{\recuadro}[1]%
{\medskip\noindent\centerline{\fbox{\parbox{.97\textwidth}{\medskip\centerline{\parbox{.95\textwidth}{#1}}\medskip}}}\medskip}

\DeclareMathOperator{\sgn}{sgn}
\DeclareMathOperator{\cotan}{cotan}
\DeclareMathOperator{\cosec}{cosec}
\DeclareMathOperator{\Dom}{Dom}
\DeclareMathOperator{\dom}{Dom}
\DeclareMathOperator{\im}{Im}

\DeclareMathOperator{\SF}{SF}
\newcommand{\SFS}{\SF^s}
\newcommand{\SFC}{\SF^c}
\DeclareMathOperator{\vol}{vol}
\DeclareMathOperator{\length}{longitud}
\newcommand{\area}{\mathrm{\protect\acute{a}rea}}
\newcommand{\dvol}{\,d\mathrm{vol}}
\newcommand{\dA}{\,dA}
\newcommand{\dx}{\,dx}
\newcommand{\dy}{\,dy}
\newcommand{\dz}{\,dz}
\newcommand{\ds}{\,ds}
\newcommand{\dr}{\,dr}
\newcommand{\dtheta}{\,d\theta}
\newcommand{\dsv}{\cdot d\ol{s}}
\newcommand{\dS}{\,d\sigma}
\newcommand{\dt}{\,dt}
\newcommand{\du}{\,du}
\newcommand{\dv}{\,dv}
\newcommand{\LL}{\mathcal{L}}
\DeclareMathOperator{\di}{div}
\DeclareMathOperator{\proy}{Proy}
\DeclareMathOperator{\Circ}{Circ}
%\usepackage[notcite,notref]{showkeys}
\newcommand{\Chi}{\raise2pt\hbox{$\chi$}}

\newcommand{\pedro}{}
\newcommand{\eps}{\varepsilon}
\newcommand{\F}{\mathcal{F}}
\newcommand{\R}{\mathbb{R}}
\newcommand{\C}{\mathbb{C}}
\newcommand{\V}{\mathbb{V}}
\newcommand{\N}{\mathbb{N}}
\newcommand{\Q}{\mathbb{Q}}
\newcommand{\Z}{\mathbb{Z}}
\newcommand{\D}{\displaystyle}
\newcommand{\gr}{^{\text{o}}}
\newcommand{\ux}{\D\frac{\partial u}{\partial x}}
\newcommand{\uy}{\D\frac{\partial u}{\partial y}}
\newcommand{\uxx}{\D\frac{\partial^2 u}{\partial x^2}}
\newcommand{\uyy}{\D\frac{\partial^2 u}{\partial y^2}}
\newcommand{\uzz}{\D\frac{\partial^2 u}{\partial z^2}}
\newcommand{\phixx}{\phi_{xx}}
\newcommand{\phiyy}{\phi_{yy}}
\newcommand{\phizz}{\phi_{zz}}
\newcommand{\phin}{\phi_n}
\newcommand{\phinm}{\phi_{n\,m}}
\newcommand{\phikl}{\phi_{k\,\ell}}
\newcommand{\lambdanm}{\lambda_{n\,m}}
\newcommand{\znm}{z_{n\,m}}
\newcommand{\anm}{a_{n\,m}}
\newcommand{\bnm}{b_{n\,m}}
\newcommand{\fnm}[1][]{f_{n\,m_{#1}}}
\newcommand{\ut}{\D\frac{\partial u}{\partial t}}
\newcommand{\vphi}{\boldsymbol{\phi}}
\newcommand{\enf}{\textbf}
\newcommand{\nn}{\ol{n}}
\newcommand{\pd}[3][]{\D\frac{\partial^{#1} {#2}}{\partial #3^{#1}}}
\newcommand{\dd}[3][]{\frac{d^{#1} {#2}}{d #3^{#1}}}
\newcommand{\grad}{\nabla}
\newcommand{\ii}{\ol{i}}
\newcommand{\jj}{\ol{j}}
\newcommand{\kk}{\ol{k}}
\DeclareMathOperator{\sop}{sop}

\newcommand{\esp}{\R^3}
\newcommand{\obs}{\Theta}
\newcommand{\ol}{\overline}
\newcommand{\ul}{\underline}
\newcommand{\oll}[1]{\ol{\ol #1}}
\newcommand{\olll}[1]{\ol{\ol{\ol #1}}}
\newcommand{\inv}{^{-1}}
\newcommand{\olnabla}{\ol\nabla}
