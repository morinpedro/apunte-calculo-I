\item Calcular las siguientes integrales indefinidas:
\begin{multicols}{2}
  \begin{enumerate}
    \item $\D \int \frac{1}{x^4} \dx$
    \item $\D \int \big(ax+b\big) \dx$
    \item $\D \int \frac{1}{\sqrt{1+x}} \dx$
    \item $\D \int \bigg(\frac{x^3-1}{x^2}\bigg) \dx$
    \item $\D \int \bigg(\sqrt{x}-\frac{1}{\sqrt{x}}\bigg) \dx$
    \item $\D \int \big(\sen x+\cos x\big) \dx$
    \item $\D \int \big(\senh x+\cosh x\big) \dx$
    \item $\D \int \frac{g'(x)}{g(x)^2} \dx$
  \end{enumerate}
\end{multicols}

\item Hallar $f$ a partir de la información dada:
\begin{multicols}{2}
  \begin{enumerate}
    \item $f'(x)=2x-1$,\ \ $f(3)= $
    \item $f'(x)=ax+b$,\ \ $f(0)=c $
    \item $f'(x)=\sen x$,\ \ $f(0)= 2$
    \item $f'(x)=\cos x$,\ \ $f(\pi )=3 $
  \end{enumerate}
\end{multicols}

\item Un objeto se mueve a lo largo de un eje de coordenadas con una velocidad de $v(t)=6t^2-6$ metros por segundo. Su posición inicial (posición en tiempo $t=0$) es 2 unidades a la izquierda del origen.
\begin{enumerate}
  \item Hallar la posición del objeto $3$ segundos más tarde.
  \item Representar esquemáticamente el movimiento del objeto como en el ejemplo~\ref{ej:velocidad}.
  \item ?`Cuál es la distancia entre la posición del objeto a tiempo $t=0$ y la posición a tiempo $t=3$?
  \item ?`Cuál es la distancia total recorrida por el objeto entre tiempo $t=0$ y $t=1$?
\end{enumerate}