\item Supongamos que $f$ y $g$ son continuas, $a<b$ y 
$\D
\int_a^b f(x)\dx > \int_a^b g(x)\dx.
$
\begin{enumerate}
  \item ?`Se deduce necesariamente que $\int_a^b \big[ f(x)-g(x)\big]\dx > 0$?
  \item ?`Se deduce necesariamente que $f(x)>g(x)$, para todo $x\in [a,b]$?
  \item ?`Se deduce necesariamente que existe $x\in[a,b]$ tal que $f(x)>g(x)$?
  \item ?`Se deduce necesariamente que $\big|\int_a^b f(x)\dx\big| > \big|\int_a^b g(x)\dx\big| $?
\end{enumerate}

\item Supongamos que $f$ es continua, $a<b$ y 
$\D
\int_a^b f(x)\dx =0.
$
\begin{enumerate}
  \item ?`Se deduce necesariamente que $f(x)=0$, para todo $x\in [a,b]$?
  \item ?`Se deduce necesariamente que existe $x\in[a,b]$ tal que $f(x)=0$?
  \item ?`Se deduce necesariamente que $\big|\int_a^b f(x)\dx\big| =0 $?
  \item ?`Se deduce necesariamente que $\int_a^b |f(x)|\dx =0 $?
\end{enumerate}

\item Consideremos la función $f(x)=\sen(x)$ en el intervalo $[0,\pi]$.
\begin{enumerate}
  \item Hallar el área comprendida entre la gráfica de $f$ y el eje $x$.
  \item Hallar la altura del rectángulo de base $\pi$ que tiene la misma área que la región calculada en el punto anterior. ?`Cómo se relaciona este número con $\frac{\int_0^\pi \sen x\dx}{\pi-0}$?
  \item Hallar un punto $c\in(0,\pi)$ tal que $f(c)(\pi-0)= \int_0^\pi \sen x\dx$.
\end{enumerate}

\item Hallar un ejemplo de un intervalo $[a,b]$, una función $f$ continua en $[a,b]$ excepto en un punto para la que no valga el Teorema del Valor Medio del Cálculo Integral (Teorema~\ref{T:TVM-integral}).