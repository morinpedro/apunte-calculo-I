\item Demostrar que si $\left(n_k\right)_{k\in\N}$ es una sucesión estrictamente creciente de números naturales, entonces $n_k \ge k$, para todo $k\in\N$.
\item * Probar que una sucesión es de Cauchy si y sólo si dado $\epsilon>0$ existe $N_0\in\N$ tal que, para $n\ge N_0$ se cumple que
\[
|a_{n+p} - a_n| < \epsilon,\quad \text{cualquiera sea $p\in\N$}.
\]

\item Probar que si una sucesión es de Cauchy, entonces, cualquiera sea $p\in\N$ se cumple que
$\D\lim_{n\to\infty} (a_{n+p}-a_n) = 0$.

\item Mostrar que la recíproca del ejercicio anterior no es cierta. Ayuda. Considerar la sucesión dada por $a_n = \ln n$, ?`Cuál es $\lim a_n$? ?`Cuál es $\D\lim_{n\to\infty} (a_{n+p}-a_n)$? 

\item Demostrar que la sucesión dada por $a_n = (-1)^n+\frac1n$ no es convergente.

\item Encontrar tres ejemplos de sucesiones acotadas que no sean convergentes.

