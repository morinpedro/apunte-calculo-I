    \item Para las siguientes sucesiones, decir cuáles son crecientes, cuáles estrictamente crecientes, cuáles decrecientes, cuáles estrictamente decrecientes, y cuáles acotadas:
    \begin{multicols}{2}
        \begin{enumerate}
        \item $\D a_n = \frac{n}{n+1}$
        \mara{\item $\D b_n = \frac{n^2}{n+1}$}
        \item $\D c_n = \frac{n!}{n^n}$
        \item $\D d_n = \frac{1}{\sqrt{n+1}-\sqrt{n}}$
        \item $\D e_n = \frac{1}{n+1} + \frac{1}{n+2} + \dots + \frac{1}{2n}$
        \mara{\item $\D f_n = \frac{2^n-1}{2^n} $}
    \end{enumerate}
    \end{multicols}
\item Sea \sucan una sucesión decreciente. Demostrar que la sucesión $(-a_n)_{\niN}$ es creciente.


\item Probar que las siguientes sucesiones son convergentes, y calcular su límite:
    \begin{enumerate}
        \item $a_1 = \sqrt3$, $\D a_{n+1} = \sqrt{3+a_n}$, \niN.
        \item $b_1 = \sqrt5$, $\D b_{n+1} = \sqrt{5+b_n}$, \niN.
        \item \mara{$c_1 = 1$, $\D c_{n+1} = 1 + \sqrt{c_n}$, \niN.}
    \end{enumerate}

