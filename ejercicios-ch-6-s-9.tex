\item Calcular las siguientes integrales impropias:
\begin{multicols}{2}
\begin{enumerate}
  \item $\D \int_1^\infty \frac1{x^2} \dx$.
  \item $\D \int_0^\infty \frac1{1+x^2} \dx$.
  \item $\D \int_0^\infty e^{p\,x}  \dx$, $p>0$.
  \item $\D \int_0^\infty e^{-p\,x}  \dx$, $p>0$.
  \item $\D \int_0^8 \frac{dx}{x^{2/3}}$.
  \item $\D \int_0^1 \frac{dx}{x^2}$.
  \item $\D \int_0^2 \frac{x}{\sqrt{4-x^2}}  \dx$.
  \item $\D \int_{-\infty}^\infty \frac1{1+x^2}  \dx$.
  \item $\D \int_{-\infty}^0 x\,e^x  \dx$.
  \item $\D \int_0^\infty e^{-x}\sen x  \dx$.
  \item $\D \int_0^\infty \cos^2 x  \dx$.
\end{enumerate}  
\end{multicols}
\item ?`Para qué valores de $\alpha>0$ es convergente la integral $\D \int_0^1 \frac1{x^\alpha}\dx$?
\item ?`Para qué valores de $\alpha>0$ es convergente la integral $\D \int_1^{+\infty} \frac1{x^\alpha}\dx$?
\item Demostrar por inducción que para todo $n\in\N_0$,
\[
\int_0^\infty x^n e^{-x}\dx = n!.
\]
\item ?`Para qué valores de $r>0$ es convergente la siguiente integral?
\[
\int_0^\infty x^r e^{-x}\dx.
\]
(Ayuda: dividir la integral entre $0$ y $1$ y entre $1$ y $\infty$ y utilizar comparación con las integrales del ejercicio anterior).
\item Sea $\Omega$ la región que se encuentra entre el eje $x$ y la gráfica de $y=e^{-x}$, para $x\in[0,+\infty)$.
\begin{enumerate}
  \item Bosquejar $\Omega$.
  \item Hallar el área de $\Omega$.
  \item Considerar el sólido de revolución que se obtiene al rotar $\Omega$ alrededor del eje $x$. Calcular su volumen.
\end{enumerate}
\item Sea $\Omega$ la región que se encuentra entre el eje $x$ y la gráfica de $y=1/x^2$, para $x\in[1,+\infty)$.
\begin{enumerate}
  \item Bosquejar $\Omega$.
  \item Hallar el área de $\Omega$.
  \item Considerar el sólido de revolución que se obtiene al rotar $\Omega$ alrededor del eje $x$. Calcular su volumen.
\end{enumerate}
\item Usar el criterio de comparación para determinar si las siguientes integrales convergen:
\begin{multicols}{2}
  \begin{enumerate}
    \item $\D \int_1^\infty \frac{x}{\sqrt{1+x^5}}\dx$
    \item $\D \int_1^\infty 2^{-x^2}\dx$
    \item $\D \int_\pi^\infty \frac{\sen^2(2x)}{x^2}\dx$
    \item $\D \int_1^\infty \frac{\ln x}{x^2}\dx$
  \end{enumerate}
  
\end{multicols}

\item Consideremos la serie armónica generalizada $\sum_{k=1}^\infty \frac1{k^p}$, para $p>1$ (que es convergente).
Usar~\eqref{eq:cota-cola-serie} para demostrar que
\[
\frac{1}{(p-1)(n+1)^{p-1}}
< \underbrace{\sum_{k=1}^\infty \frac1{k^p}}_S-\underbrace{\sum_{k=1}^n \frac1{k^p}}_{S_n}
< \frac{1}{(p-1)n^{p-1}}.
\]
Este resultado proporciona cotas para el \emph{error} $e_n$ que se obtiene al sumar $S_n$ para aproximar la suma $S$ de la serie armónica generalizada.

\item Consideremos la serie armónica generalizada $\sum_{k=1}^\infty \frac1{k^2}$.
\begin{enumerate}
  \item Si se desea utilizar $S_{100}$ para aproximar $\sum_{k=1}^\infty \frac1{k^2}$. ?`Cuáles serían las cotas del error cometido?
  \item ?`Qué valor de $n$ habría que elegir para asegurar qu $|S-S_n| $ sea menor que $0.0001$?
\end{enumerate}

\item Consideremos la serie armónica generalizada $\sum_{k=1}^\infty \frac1{k^3}$.
\begin{enumerate}
  \item Si se desea utilizar $S_{100}$ para aproximar $\sum_{k=1}^\infty \frac1{k^3}$. ?`Cuáles serían las cotas del error cometido?
  \item ?`Qué valor de $n$ habría que elegir para asegurar qu $|S-S_n| $ sea menor que $0.0001$?
\end{enumerate}