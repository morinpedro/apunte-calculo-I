\item Para las siguientes funciones, determinar cuáles son acotadas superiormente e inferiormente en el intervalo indicado y si alcanzan su máximo y mínimo en algún punto del intervalo:

\begin{multicols}{2}
\begin{enumerate}
    \item $f(x)= x^2$ en $(-1,1)$
    \item $f(x)= x^2$ en $[-1,1]$
    \item $f(x)= x^3$ en $(-1,1)$
    \item $f(x)= x^3$ en $[-1,1]$
    \item $\D f(x)= \frac1x$ en $(0,1)$
    \item $\D f(x)= \frac1x$ en $(0,1]$
    \item $f(x)= \sqrt{x}$ en $(0,1)$
    \item $f(x)= \sqrt{x}$ en $[0,1]$
\end{enumerate}   
\end{multicols}



\item Probar que la ecuación $x^3-3x+1=0$ tiene una raíz real en el intervalo $(1,2)$.

\item Probar que existe $x\in\R$ tal que $\cos x=x$.

\item Probar que si $p(x)$ es un polinomio \emph{mónico} de grado impar, entonces existe $x_0\in\R$ tal que $p(x_0)=0$.

\item Probar que si $p(x)$ es un polinomio de grado impar, entonces existe $x_0\in\R$ tal que $p(x_0)=0$.

\item Probar que si $n$ es un número natural, entonces en el intervalo $\D\Big(n\pi-\frac\pi2,n\pi+\frac\pi2\Big)$ hay un $x$ que cumple $\tan x=-x$. Ayudarse con GeoGebra, graficar las funciones $\tan x$ y $-x$ para ver cómo usar el Teorema~\ref{TVI} para demostrar lo pedido.

\item Demostrar que existe un número real $x$ tal que
\[
x^{179} + \frac{163}{1+x^2+\sen^2 x} = 119.
\]

\item Supongamos que $f$ y $g$ son continuas en $\R$ y que $f(a)>g(a)$ para algún $a\in\R$ y $f(b)<g(b)$ para algún $b\in\R$. Probar que existe $x\in\R$ tal que $f(x)=g(x)$.

\item Supongamos que $f:\R\to\R$ es continua y que $f(x)\in\Q$, para todo $x\in\R$. Probar que entonces $f$ es constante.
