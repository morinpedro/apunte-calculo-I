\item Probar que la sucesión \emph{constante} $a_n=a$ para todo $n\in\N$, tiene límite $a$.

\item Probar que si $\lim a_n = a$, entonces $\lim |a_n| = |a|$ (ayuda, usar la desigualdad triangular del ejercicio~\ref{ej:triangular resta} del Capítulo~\ref{Cap:Reales}).

\item Probar las siguientes afirmaciones:

\begin{multicols}{2}
    \begin{enumerate}
        \item $\D \lim \frac1{\sqrt{n}} = 0 $
        \item $\D \lim \frac{1}{\sqrt{n+1}+\sqrt{n}} = 0 $
        \item $\D \lim \frac{(-1)^n}{3n^2-4n} = 0 $
        \item $\D \lim \frac{(-1)^{n-1}}{2-n^2} = 0 $
        \item $\D \lim \frac{n}{n+1} = 1 $
        \item $\D \lim \frac{3n}{4n+2} = \frac34 $
        \item $\D \lim \frac{2n+3}{n^2-2n-3} = 0 $
        \item $\D \lim \frac{-3n+1}{4n^2-3n+4} = 0 $
        \item $\D \lim \frac{3n^2+2n-2}{n^2+1} = 3 $
        \item $\D \lim \frac{2n^2-3n+1}{3n^2+2n-1} = \frac23 $
        \item $\D \lim \frac{n^2+n+1000}{3n^2-14n-7} = \frac13 $
        % \item $\D \lim \frac{n}{n^{3/2}+1} = 0 $
        % \item $\D \lim \frac{n^{2/3}+100}{n^{3/4}+4} = 0 $
        % \item $\D \lim \frac{3n^{2/3}+n^{4/5}+2n^{5/2}}{n^3+n^{2/3}+5n} = 0 $
        % \item $\D \lim \frac{2n^{3/4}+\sqrt n}{n^{3/4}} = 2 $
        % \item $\D \lim \frac{n^2-3n^{7/2}+20}{n^{7/2}-6n^3+3n^2-2n} = -3$
    \end{enumerate}
\end{multicols}

\item Probar que $\D \lim \big( \sqrt{n+1} - \sqrt{n}\big) = 0$.
(Sugerencia: multiplicar y dividir por \emph{el conjugado} $\sqrt{n+1} + \sqrt{n}$)

\item Sea $\big( a_n \big)_{n\in\N}$ una sucesión convergente con límite $\ell$. Probar que si $\big(b_n \big)_{n\in\N}$ está definida por 
\[
b_n = a_{n+1}, \qquad \text{o sea $b_1=a_2$, $b_2=a_3$, $b_3=a_4$, \dots},
\]
entonces $\D\lim b_n = \ell$.

\item Sea $\big( a_n \big)_{n\in\N}$ una sucesión convergente con límite $\ell$, y sea $p\in\N$.
Probar que si $\big(b_n \big)_{n\in\N}$ está definida por 
\[
b_n = a_{n+p}, \qquad \text{o sea $b_1=a_{p+1}$, $b_2=a_{p+2}$, $b_3=a_{p+3}$, \dots},
\]
entonces $\D\lim b_n = \ell$.

\item Sea $\big( a_n \big)_{n\in\N}$ una sucesión convergente con límite $\ell$, y sea $p\in\N$.
Probar que si $\big(b_n \big)_{n\in\N}$ está definida por 
\[
\begin{cases}
    b_1 &= \text{cualquier cosa},\\
b_2 &= \text{cualquier cosa},\\
 &\vdots\\
b_p &= \text{cualquier cosa},
\end{cases}
\qquad\qquad \text{y }\quad b_k = a_k, \ \text{ para } \ k>p,
\]
entonces $\D\lim b_n = \ell$.
