\item Consideremos la función constante $f(x)=a$, para algún $a\in\R$. Demostrar usando la definición que $f$ es continua en $x_0$ para todo $x_0\in\R$.
\item Consideremos la función \emph{identidad} $f(x)=x$. Demostrar usando la definición que $f$ es continua en $x_0$ para todo $x_0\in\R$.
\item Demostrar que si $a$ y $b$ son números reales, entonces la función lineal $f(x)=ax+b$ es continua en $x_0$ para todo $x_0\in\R$.
Usar lo demostrado en los ejercicios anteriores y las propiedades vistas en esta sección.
\item Demostrar que si $a$, $b$ y $c$ son números reales, entonces la función cuadrática $f(x)=ax^2+bx+c$ es continua en $x_0$ para todo $x_0\in\R$.
Usar lo demostrado en los ejercicios anteriores y las propiedades vistas en esta sección.
\item\label{ej:polinomios-continuos} Demostrar que si $p(x)$ es una función polinomial, entonces $p$ es continua en $x_0$ para todo $x_0\in\R$ (Ayuda: usar inducción sobre el grado polinomial).
\item Demostrar que si $p(x)$ y $q(x)$ son funciones polinomiales, entonces la función racional $\frac pq$ es continua en $x_0$ para todo $x_0\in\R$, excepto para aquellos puntos donde $q(x)$ se anula. 

