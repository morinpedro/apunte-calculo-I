
\chapter{Integración}

En este capítulo vamos a estudiar el concepto que, junto con el de derivada, forma el núcleo del cálculo diferencial e integral.
Veremos que dicho concepto, el de \emph{integral}, será definido por motivos totalmente ajenos a los que nos llevaron a definir la derivada, pero también veremos que está íntimamente conectado a través del teorema fundamental del cálculo.

\section{La integral definida de una función continua}
% Seguimos el Salas-Hille

La noción de integral definida como la presentaremos aquí aparece naturalmente cuando se intenta resolver el problema del cálculo del área de una región plana. El lector recordará la fórmula del área de ciertas figuras planas, por ejemplo el rectángulo el triángulo, el círculo, etc.

\centerline{\includegraphics[width=.8\textwidth]{pics/areas-simples.png}}

En este capítulo veremos cómo calcular el área de figuras más generales. Comenzaremos con el cálculo del área que queda delimitada de la siguiente manera:

\noindent
\begin{minipage}{.5\textwidth}
  \begin{itemize}
  \item Por debajo, por el eje $x$;
  \item Por encima, por la gráfica de $y=f(x)$;
  \item A la izquierda por la recta $x=a$; 
  \item A la derecha por la recta $x=b$.
\end{itemize}
\end{minipage}
\begin{minipage}{.5\textwidth}
  \centering
  \includegraphics[width=.9\textwidth]{pics/area-bajo-curva.png}
\end{minipage}

\noindent
\begin{minipage}{.5\textwidth}
  Consideramos ahora una subdivisión del intervalo $[a,b]$ en un número finito de subintervalos que no se solapan:
  \[
  [x_0,x_1],\ [x_1,x_2],\dots,\ [x_{n-1},x_n]
  \]
  con $\D a=x_0<x_1<\dots<x_n=b$.

  Observamos que el área de la región $\Omega$ es lo mismo que la suma de las áreas de las regiones $\Omega_i$.
\end{minipage}
\begin{minipage}{.5\textwidth}
  \centering
  \includegraphics[width=.9\textwidth]{pics/area-bajo-curva-particionada.png}
\end{minipage}

Podemos aproximar el área total de $\Omega$ aproximando el área de cada una de las subregiones $\Omega_i$ y sumando los resultados.
Definimos, para cada $i=1,2,\dots,n$ las siguientes cantidades:
\[ 
  m_i = \min_{x\in [x_{i-1},x_i]}f(x)=\min_{[x_{i-1},x_i]}f,
  \qquad
  M_i = \max_{x\in [x_{i-1},x_i]}f(x)=\max_{[x_{i-1},x_i]}f.
\]
Luego, si $r_i = [x_{i-1},x_i]\times [0,m_i]$ y $R_i = [x_{i-1},x_i]\times [0,M_i]$, resulta que
\[
r_i \subset \Omega_i \subset R_i
\quad\text{y entonces}\quad
\area(r_i)\le \area(\Omega_i) \le \area(R_i).
\]

\centerline{
  \includegraphics[width=.9\textwidth]{pics/area-omega-i.png}
}

Dado que el área de un rectángulo es el producto de la base por la altura, obtenemos
\[
m_i\, (x_i-x_i-1)
\le \area(\Omega_i) \le
M_i\, (x_i-x_i-1),
\qquad\text{para $i=1,2,\dots,n$}.
\]
Definiendo $\Delta x_i=x_i-x_{i-1}$ tenemos que 
\[
m_i\, \Delta x_i
\le \area(\Omega_i) \le
M_i\, \Delta x_i,
\qquad\text{para $i=1,2,\dots,n$}.
\]
Sumando, obtenemos
\begin{equation}
  \label{eq:L<I<U}
m_1 \Delta x_1 + m_2 \Delta x_2 + \dots + m_n \Delta x_N
\le \area(\Omega_i) \le
M_1 \Delta x_1 + M_2 \Delta x_2 + \dots + M_n \Delta x_N.
\end{equation}
La suma $m_1 \Delta x_1 + m_2 \Delta x_2 + \dots + m_n \Delta x_N$ es una \emph{suma inferior de $f$} 
y la suma $M_1 \Delta x_1 + M_2 \Delta x_2 + \dots + M_n \Delta x_N$ es una \emph{suma superior de $f$}.

\centerline{
  \includegraphics[width=.9\textwidth]{pics/sumas-superiores-inferiores.png}
}

La desigualdad~\eqref{eq:L<I<U} nos dice que para que un número pueda ser candidato al título de \emph{área de $\Omega$}, tal número ha de ser mayor o igual que cualquier suma inferior de $f$ y menor o igual que cualquier suma superior de $f$. Puede demostrarse, aunque no lo veremos aquí, que si $f$ es una función continua en $[a,b]$ entonces existe un número y sólo uno que cumple estas condiciones. A tal número lo llamaremos \emph{área de $\Omega$}.

\subsection*{Integral definida}

El procedimiento seguido para resolver el problema de área se llama \emph{integración} y el resultado final de este procedimiento se denomina \emph{integral definida}. A continuación estableceremos estas nociones con más precisión.

\begin{definition}
  Llamaremos \emph{partición} del intervalo cerrado $[a,b]$ a todo subconjunto finito de $[a,b]$ que contenga a los extremos. Es conveniente identificar los elementos de la partición con un índice acorde al orden narural. Así, cuando escribimos que 
  \[
  P=\{x_0,x_1, \dots ,x_n\} \text{ es una partición de $[a,b]$},
  \]
  entendemos que $\D a=x_0<x_1<\dots<x_n=b$.
\end{definition}

Supongamos ahora que $f$ es una función continua sobre $[a,b]$. En cada subintervalo $[x_{i-1},x_i]$ la función $f$ toma entonces un valor máximo $M_i$ y un mínimo $m_i$. Definimos entonces:

\begin{definition}
  Sea $f$ una función continua sobre $[a,b]$ y sea $P=\{a=x_0<x_1<\dots<x_n=b\}$ una partición de $[a,b]$. Luego definimos, para cada $i=1,2,\dots,n$:
  \[ 
    m_i = \min_{[x_{i-1},x_i]}f,
    \qquad
    M_i = \max_{[x_{i-1},x_i]}f.
  \]
  El número 
  \[
  L(P,f)= m_1 \Delta x_1 + m_2 \Delta x_2 + \dots + m_n \Delta x_N 
  = \sum_{i=1}^n m_i \Delta x_i,
  \]
  se denomina
  \emph{suma inferior de $f$ correspondiente a la partición $P$};
  y el número 
  \[
  U(P,f)= M_1 \Delta x_1 + M_2 \Delta x_2 + \dots + M_n \Delta x_N
  = \sum_{i=1}^n M_i \Delta x_i,
  \]
  se denomina \emph{suma superior de $f$ correspondiente a la partición $P$}.
\end{definition}

\begin{theorem}
  Si $f$ es una función continua en $[a,b]$ existe un único número $I$ que satisface la desigualdad 
  \[
  L(P,f)\le I\le U(P,f),
  \qquad\text{para toda partición $P$ de $[a,b]$}
  \]
\end{theorem}

\begin{proof}
  La demostración de este teorema queda fuera del alcance de este apunte, y será discutida en las clases de Coloquio de Demostraciones.
\end{proof}

A partir de este último teorema llegamos a la definición principal de esta sección:

\begin{definition}[Integral definida]
  Sea $f$ una función continua en $[a,b]$. El único número $I$ que satisface 
  \[
    L(P,f)\le I\le U(P,f),
    \qquad\text{para toda partición $P$ de $[a,b]$}
  \]
  se llama \emph{integral definida} (o simplemente \emph{integral}) de $f$ entre $a$ y $b$ y se designa por
  \[
  \int_a^b f(x)\dx.
  \]
\end{definition}

El símbolo $\int$ fue introducido por Leibniz y se llama \emph{signo integral}. En realidad es una \emph{S} (de \emph{suma}) estirada. Los números $a$ y $b$ se denominan \emph{límites de integración} ($a$ es el límite inferior y $b$ el límite superior). La función a integrar se llama \emph{integrando}. En algunos libros suele escribirse más brevemente $\int_a^b f$, omitiendo la variable $x$.

En la expresión $\int_a^b f(x)\dx$ la variable $x$ es una \emph{variable muda}. Esto quiere decir que puede ser sustituida por cualquier otra letra no utilizada hasta el momento. Así:
\[
\int_a^b f(x)\, dx
= \int_a^b f(t)\, dt
= \int_a^b f(u)\, du
= \int_a^b f(z)\, dz.
\]

En la introducción a este capítulo comenzamos con una aplicación inmediata de la integral definida:
Si $f$ es no negativa en $[a,b]$, entonces
\[
A = \int_a^b f(x)\dx
\]
da como resultado el área debajo de la gráfica de $f$.
Volveremos más adelante sobre esta aplicación.
Ahora veamos algunos ejemplos sencillos que nos ayudarán a comprender la definición.

\begin{example}
  Si $f(x)=c$, una constante, para todo $x\in[a,b]$, ?`cuánto vale $\int_a^b f(x)\dx$?

  Veamos, si $P=\{a=x_0<x_1<\dots<x_n=b\}$ es una partición de $[a,b]$, tenemos que 
\[ 
    m_i = \min_{[x_{i-1},x_i]}f = c
    \qquad
    M_i = \max_{[x_{i-1},x_i]}f = c;
  \]
  por lo que 
  \[
  L(P,f)
  = \sum_{i=1}^n m_i \Delta x_i
  = \sum_{i=1}^n c \Delta x_i
  = c \sum_{i=1}^n \Delta x_i
  = c (b-a)
  \]
  y 
  \[
  U(P,f)
  = \sum_{i=1}^n M_i \Delta x_i
  = \sum_{i=1}^n c \Delta x_i
  = c \sum_{i=1}^n \Delta x_i
  = c (b-a).
  \]
  Luego, 
  \[
    L(P,f)\le c(b-a)\le U(P,f),
    \qquad\text{para toda partición $P$ de $[a,b]$},
  \]
  y concluimos que 
  \[
  \int_a^b f(x)\dx = \int_a^b c \dx = c (b-a).
  \]

  Como ejemplos concretos:
  \begin{align*}
    \int_{-1}^1 3 \dx = 3 \big(1 - (-1)\big) = 3 \cdot 2 = 6;
    % \qquad\text{y}
    \\
    \int_{4}^{10} -3 \dx = -2 \big(10 - 4\big) = -2 \cdot 6 = -12.
  \end{align*}
\end{example}

\begin{example}
  Nos proponemos estudiar ahora la integral de la función $f(x)=x$, es decir $\int_a^b x\dx$.
  Veamos, si $P=\{a=x_0<x_1<\dots<x_n=b\}$ es una partición de $[a,b]$, tenemos que 
  \[ 
      m_i = \min_{[x_{i-1},x_i]}f = x_{i-1}
      \qquad
      M_i = \max_{[x_{i-1},x_i]}f = x_i;
    \]
    por lo que 
    \[
    L(P,f)
    = \sum_{i=1}^n m_i \Delta x_i
    = \sum_{i=1}^n x_{i-1} (x_{i}-x_{i-1})
    \]
    y 
    \[
    U(P,f)
    = \sum_{i=1}^n M_i \Delta x_i
    = \sum_{i=1}^n x_{i} (x_{i}-x_{i-1}).
    \]
    Observamos que para cada índice $i=1,2,\dots,n$, 
    \[
    x_{i-1}\le \frac12 (x_i+x_{i-1})\le x_i.
    \]
    Luego, multiplicando por $\Delta x_i = x_i-x_{i-1}>0$, tenemos que
    \[
    x_{i-1} \Delta x_i 
    \le \frac12 (x_i+x_{i-1})(x_i-x_{i-1}) 
    = \frac12 (x_i^2-x_{i-1}^2) = 
    \le x_i \Delta x_i.
    \]
    Sumando para $i$ de $1$ a $n$ obtenemos
    \[
    L(P,f) \le \sum_{i=1}^n \frac12 (x_i^2-x_{i-1}^2) \le U(P,f).
    \]
    Pero 
    \begin{align*}
      \sum_{i=1}^n \frac12 (x_i^2-x_{i-1}^2)
      &= \frac12 \Big[ (x_1^2-x_0^2) + (x_2^2-x_1^2) + \dots + (x_n^2-x_{n-1}^2)  \Big]
      \\
      &= \frac12 \big(x_n^2-x_0^2\big)
      = \frac12 \big(b^2-a^2\big).
    \end{align*}
    Luego, 
    \[
      L(P,f)\le \frac12 \big(b^2-a^2\big) \le U(P,f),
      \qquad\text{para toda partición $P$ de $[a,b]$},
    \]
    y por lo tanto
    \[
    \int_a^b x\dx = \frac12 \big(b^2-a^2\big).
    \]
    Como ejemplos concretos:
    \begin{align*}
      \int_{-1}^1 x \dx = \frac12 \big(3^2 - (-1)^2\big) = \frac12 \cdot 8 = 4;
      % \qquad\text{y}
      \\
      \int_{-2}^{2} x \dx = \frac12 \big(2^2 - (-2)^2\big) = \frac12 \cdot 0 = 0.
    \end{align*}\end{example}

\begin{example}
  Consideremos un último ejemplo, de una función \emph{discontinua}, pero \emph{continua a trozos}.
  Sea 
  \[
  f(x)=\begin{cases}
    0, \quad&\text{si $x\neq 1$},
    \\
    1, \quad&\text{si $x=1$}.
  \end{cases}
  \]
  Afirmamos que $\int_0^2 f(x)\dx=0$, es decir, cambiar el valor de la función en un punto no afecta el valor de la integral.
  Claramente, cualquiera sea la partición $P$ del intervalo $[0,2]$, se tiene que 
  \[
  0=L(P,f) \le 0 \le U(P,f).
  \]
  Veamos que $0$ es el único número que cumple dicha desigualdad para toda partición.
  Consideremos la partición $P_\epsilon = \{0, 1-\epsilon, 1+\epsilon, 2\}$, para $0<\epsilon<1$.
  Luego, $L(P_\epsilon,f)=0$ y 
  \[
  U(P,f)=0 (x_1-x_0) + 1 (x_2-x_1) + 0 (x_3-x_2)
  = (x_2-x_1)= \big((1+\epsilon)-(1-\epsilon)\big)=2\epsilon.
  \]
  El único número que cumple $0\le I \le 2\epsilon$ para todo $\epsilon\in(0,1)$ es $I=0$.
  Por lo tanto $\int_0^2 f(x)\dx=0$.
\end{example}


\subsubsection*{Ejercicios de la sección~\getcurrentref{chapter}.\getcurrentref{section}}

\begin{enumerate}
\item Hallar $L(P,f)$ y $U(P,f)$ en cada uno de los siguientes casos:
\begin{enumerate}
  \item $\D f(x) = 1-x$, $P=\{0, \frac13, \frac34, 1, 2\}$
  \item $\D f(x) = \sqrt{x}$, $P=\{0, \frac1{25}, \frac4{25}, \frac9{25}, \frac{16}{25}, 1\}$
  \item $\D f(x) = x^2$, $P=\{-1,-\frac12, -\frac14, 0, \frac14, \frac12, 1\}$
\end{enumerate}

\item Sea $f$ una función continua en $[-1,1]$. Explicar por qué cada una de las siguientes afirmaciones es falsa.
\begin{enumerate}
  \item $L(P,f)=3$ y $U(P,f)=2$.
  \item $L(P,f)=3$ y $U(P,f)=6$ y $\int_{-1}^1 f(x)\dx=2$.
  \item $L(P,f)=3$ y $U(P,f)=6$ y $\int_{-1}^1 f(x)\dx=10$.
\end{enumerate}

\item Sea $f:[0,1]\to \R$ definida por
\[
\begin{cases}
  1, \quad&\text{si $x\in\Q$,}
 \\ 
 0, \quad&\text{si $x\notin\Q$.}
\end{cases}
\]
Si $P$ es una partición de $[0,1]$.
?`Cuánto valen $L(P,f)$ y $U(P,f)$?
?`Cuántos números $I$ hay que cumplen $L(P,f) \le I \le U(P,f)$, para toda partición $P$?
\end{enumerate}


\section{El Teorema Fundamental del Cálculo}
  %La función \texorpdfstring{$F(x)=\int_a^x f(t)\dx$}{integral hasta \emph{x}}}

En los ejemplos de la sección anterior tuvimos que confiar en que se nos ocurriera una idea para saber cómo encontrar el número $I$ que da la integral. En esta sección estudiaremos propiedades de la integral que nos permitirán encontrar una forma más sistemática y sencilla de calcular integrales.

Para ello necesitamos primero la siguiente Proposición.

\begin{proposition}
  Sea $f$ es continua en $[a,b]$, y sean $P$ y $Q$ dos particiones del intervalo $[a,b]$.
  Si $P\subset Q$, entonces 
  \[
  L(P,f) \le L(Q,f)
  \qquad\text{y}\qquad
  U(Q,f) \le U(P,f).
  \]
  Es decir, al agregar puntos a una partición, \emph{las sumas inferiores crecen} y \emph{las sumas superiores decrecen}.
\end{proposition}

\begin{proof}
  Basta con probar el caso en que $Q$ tiene un punto más que $P$, luego proceder por inducción.
  Supongamos entonces que $P=\{a=x_0<x_2<\dots<x_n=b\}$ y que $Q = P\cup \{p\}$. Sea $i_0\in\{x_1,x_2,\dots,x_n\}$ tal que $x_{i_0-1}<p<x_{i_0}$.
  Gráficamente, la situación en ese intervalo se vería así:

  \centerline{\includegraphics[width=.8\textwidth]{pics/sumas-insertar-punto.png}}

  Por lo tanto, el único término de la suma inferior de $P$ que se cambia, cambia por dos términos que al sumarlos dan mayor o igual al término original.
  Y el único término de la suma superior de $P$ que se cambia, cambia por dos términos que al sumarlos dan menor o igual al término original. 
  De aquí concluimos la afirmación de la proposición.
\end{proof}

El siguiente teorema es bastante obvio si pensamos en la interpretación geométrica de la integral como un área.

\begin{theorem}
  Si $f$ es continua en $[a,b]$ y $a<c<b$, entonces
  \[
  \int_a^b f(x)\dx = \int_a^c f(x)\dx + \int_c^b f(x)\dx.
  \]
\end{theorem}

\noindent
\begin{minipage}{.6\textwidth}
  \[ \area(I)+\area(II)=\area(\text{región completa})\]
\end{minipage}
\begin{minipage}{.39\textwidth}
\centerline{\includegraphics[width=.8\textwidth]{pics/integral-suma-intervalos.png}}
\end{minipage}

\begin{proof}
  Para demostrar el teorema basta con que demostremos que para toda partición $P$ de $[a,b]$ se verifica que
  \[
  L(P,f) \le \int_a^c f(x)\dx + \int_c^b f(x)\dx \le U(P,f).
  \]
  Consideremos entonces una partición arbitraria $P$ de $[a,b]$ y sea $Q=P\cup\{c\}\supset P$.
  Por la proposición anterior 
  \[
  L(P,f) \le L(Q,f)
  \qquad\text{y}\qquad
  U(Q,f) \le U(P,f).
  \]
  Ahora consideramos subparticiones de $Q$ en los intervalos $[a,c]$ y $[c,b]$:
  \[
  Q_1 = Q\cap[a,c]\qquad\text y\qquad 
  Q_2 = Q\cap[c,b].
  \]
  Luego $Q_1$ y $Q_2$ son particiones de $[a,c]$ y de $[c,b]$ respectivamente, por lo tanto:
  \[
  L(Q_1,f)\le \int_a^c f(x)\dx \le U(Q_1,f)
  \qquad\text{y}\qquad
  L(Q_2,f)\le \int_c^b f(x)\dx \le U(Q_2,f).
  \]
  Sumando estas desigualdades obtenemos
  \[
  L(Q_1,f)+L(Q_2,f)\le \int_a^c f(x)\dx + \int_c^b f(x)\dx \le U(Q_1,f)+U(Q_2,f).
  \]
  Finalmente observamos que 
  \[
    L(Q_1,f)+L(Q_2,f) = L(Q,f)
    \qquad\text y\qquad
    U(Q_1,f)+U(Q_2,f) = U(Q,f),
  \]
  que implica lo que queríamos demostrar.
  \end{proof}

  Hasta ahora sólo hemos definido integrales \emph{de izquierda a derecha}: desde un número $a$ hasta un número $b$, con $b>a$. También puede integrarse en sentido opuesto, a través de la siguiente definición:
  \[
  \int_b^a f(x)\dx 
  = -\int_a^b f(x)\dx .
  \]
  Además, la integral cuando los extremos de integración coinciden se define como cero:
  \[
  \int_c^c f(x)\dx = 0.
  \]
  De esta manera, la siguiente condición se cumple independientemente del orden en que se encuentren $a$, $b$ y $c$:
  \[
    \int_a^c f(x)\dx + \int_c^b f(x)\dx=\int_a^b f(x)\dx .
  \]

\noindent
\begin{minipage}{.4\textwidth}  Dada una función continua $f$, definimos 
  \[
  F(x)=\int_a^x f(t)\dt.
  \]
  Es decir, $F(x)$ es el área de la región sombreada en las figuras, que varía al variar $x$.
\end{minipage}
\begin{minipage}{.6\textwidth}
  \begin{center}
    \includegraphics[width=.95\textwidth]{pics/integral-hasta-x.png}
  \end{center}
\end{minipage}

El siguiente teorema se conoce como Teorema Fundamental del Cálculo, y relaciona el concepto nuevo de integral con el de derivada, de la siguiente manera:

\begin{theorem}\label{T:TFC}
  Si $f$ es continua en $[a,b]$ y definimos $F:[a,b]\to\R$ como 
  \[
  F(x)=\int_a^x f(t)\dt,
  \]
  Entonces $F$ es continua en $[a,b]$ y diferenciable en $(a,b)$ y más aún, 
  \[
  F'(x)=f(x),\quad\text{para todo $x\in(a,b)$.}
  \]
\end{theorem}

\begin{proof}
  Comenzamos considerando $x\in[a,b)$ y veremos que 
  \[\lim_{h\to 0^+}\frac{F(x+h)-F(x)}{h}=f(x).\]
  Si $x<x+h\le b$, entonces
  \[F(x+h)-F(x)=\int_a^{x+h}f(t)\dt - \int_a^x f(t)\dt = \int_x^{x+h}f(t)\dt.\]
  Definimos ahora 
  $$ M_h = \max_{[x,x+h]}f\qquad\text y\qquad
  m_h = \min_{[x,x+h]}f.$$
  Entonces, 
  $$ M_h \cdot \big( (x+h)-x \big) = M_h \cdot h $$
  es una suma superior para $f$ en $[x,x+h]$ y
  $$ m_h \cdot \big( (x+h)-x \big) = m_h \cdot h $$ 
  es una suma inferior para $f$ en $[x,x+h]$.
  Luego
  $$ m_h \cdot h \le \int_x^{x+h}f(t)\dt \le M_h\cdot h,$$
  que a su vez implica
  $$ m_h \le \frac{\int_x^{x+h}f(t)\dt}h \le M_h.$$
  Es decir,
  $$ m_h \le \frac{F(x+h)-F(x)}h \le M_h.$$
  Como $f$ es continua en $[a,b]$
  \[
  \lim_{h\to0^+} m_h = f(x) = \lim_{h\to 0^+} M_h,
  \]
  y por el teorema del emparedado
  \begin{equation}\label{eq:TFC-der}
    \lim_{h\to 0^+} \frac{F(x+h)-F(x)}h = f(x).
  \end{equation}

  De manera análoga se puede demostrar que, para $x\in(a,b]$ se verifica
  \begin{equation}\label{eq:TFC-izq}
    \lim_{h\to 0^-} \frac{F(x+h)-F(x)}h = f(x).
  \end{equation}
  Esto implica que para $x\in(a,b)$ se cumplen~\eqref{eq:TFC-der} y~\eqref{eq:TFC-der} a la vez, 
  \[
  F'(x)= \lim_{h\to 0} \frac{F(x+h)-F(x)}h = f(x).
  \]
  Esto muestra que $F(x)$ es diferenciable en $(a,b)$ y su derivada es $F'(x)=f(x)$.

  Solo queda probar que $F$ es continua por derecha en $a$ y continua por izquierda en $b$.
  A partir de~\eqref{eq:TFC-der} en $a$ obtenemos
  $$ 
  \lim_{h\to 0^+} \frac{F(a+h)-F(a)}h = f(a),
  $$ 
  Que a su vez implica que
  $$ 
  \lim_{h\to 0^+} \big(F(a+h)-F(a)\big)
  = \lim_{h\to 0^+}\Big( h\cdot \frac{F(a+h)-F(a)}h \Big)= f(a)\cdot 0 = 0.
  $$ 
  Por lo tanto $\lim_{h\to 0^+} F(a+h)=F(a)$ y $F$ es continua por derecha en $a$.
  Análogamente se puede demostrar que $F$ es continua por izquierda en $b$, usando~\eqref{eq:TFC-izq}.
\end{proof}

\begin{example}
  Si $F$ está definida por 
  \[
  F(x)=\int_{-1}^x (2t+t^2) \dt,
  \qquad \text{para $-1\le x\le 5$},
  \]
  Entonces $F'(x) = 2x+x^2$.
\end{example}


\begin{example}
  Si $F$ está definida por 
  \[
  F(x)=\int_{0}^x \sen(\pi t) \dt,
  \qquad \text{para $x\in\R$},
  \]
  Entonces $F'(x) = \sen(\pi x)$.
\end{example}

A continuación veremos cómo este teorema nos sirve para calcular integrales definidas de un gran número de funciones. Para ello hacemos primero la siguiente definición.

\begin{definition}[Antiderivada]
  Sea $f$ una función continua en $[a,b]$. Una función $G$ se llama \emph{antiderivada} de $f$ en $[a,b]$ si y sólo si
  \[
  G\text{ es continua en $[a,b]$ \quad y \quad}
  G'(x)=f(x) \text{ para todo $x\in(a,b)$.}\]
\end{definition}

El Teorema~\ref{T:TFC} nos dice que, si $f$ es continua en $[a,b]$, entonces la función $F(x)$ definida por 
\[
  F(x)=\int_a^x f(t)\dt.
\]
es una antiderivada de $f$.
A partir del \emph{Teorema Fundamental del Cálculo} llegamos a un método que nos permite calcular integrales. Podemos calcular $\D\int_a^b f(t)\dt$ hallando una antiderivada de $f$.

\begin{theorem}[Regla de Barrow]
   Sea $f$ una función continua en $[a,b]$. Si $G$ es una antiderivada de $f$ en $[a,b]$, entonces
   $$ \int_a^b f(t)\dt = G(b)-G(a). $$
\end{theorem}

\begin{proof}
  Consideremos la función $F:[a,b]\to \R$ definida por $\D  F(x)=\int_a^x f(t)\dt$. 
  Luego, por un lado, 
  $$F(a)=\int_a^a f(t)\dt = 0,\qquad\text y\qquad  F(b)=\int_a^b f(t)\dt.$$

  Por otro lado, por el Teorema~\ref{T:TFC},
  $F(x)$ es una antiderivada de $f$, al igual que $G$. Luego,
  \[
  F'(x)=f(x)\quad\text y\quad G'(x)=f(x),\qquad\text{para todo $x\in(a,b)$.}
  \]
  Por el Corolario~\ref{C:f=g+k}, resulta que existe una constante $k\in\R$ tal que 
  \[
  F(x) = G(x) + k,
  \qquad\text{para todo $x\in[a,b]$.}
  \]
  Como $F(a)=0$, resulta que $G(a)+k=0$, es decir, $k = -G(a)$.
  Por lo tanto
  \[
  F(x) = G(x) - G(a),
  \qquad\text{para todo $x\in[a,b]$.}
  \]
  En particular, 
  \[
  F(b) = \int_a^b f(t)\dt = G(b)-G(a),
  \]
  que es lo que se quería demostrar.
\end{proof}

De este último teorema, obtenemos un procedimiento para calcular integrales.
\begin{quote}
  Para calcular $\D\int_a^b f(t)\dt$, debemos hacer lo siguiente:
  \begin{enumerate}
    \item Hallar una antiderivada $G$ de $f$.
    \item Calcular $G(b)-G(a)$. Ese resultado es $\D\int_a^b f(t)\dt$.
    \item FIN.
  \end{enumerate}
\end{quote}


\begin{example}
  Calcular $\D \int_1^4 x^2 \dx$.

  Debemos hallar una antiderivada de $f(x)=x^2$. Inspeccionando la tabla de derivadas, vemos que $\dd[]{x^3}{x} = 3 \,x^2$. Luego, $\D\dd{}{x}\frac{x^3}3 = x^2$ y por lo tanto $G(x)=\D\frac{x^3}3 $ es una antiderivada de $f(x)=x^2$. Luego, aplicando el teorema fundamental del Cálculo,
  \[
    \int_1^4 x^2 \dx = G(4) - G(1) = \frac{4^3}3-\frac{1^3}3 = \frac{64}3-\frac13=21.
  \]
\end{example}

\begin{example}
  Consideremos ahora el cálculo de $\D\int_0^{\pi/2} \sen x \dx$.

  Tenemos que hallar una antiderivada de $\sen x$.
  Inspeccionando nuestra memoria, recordamos que $\dd{\cos x}{x}=-\sen x$, luego $-\cos x$ es una antiderivada de $\sen x$. Entonces
  \[
  \int_0^{\pi/2} \sen x\dx = (-\cos \frac\pi2)-(-\cos 0)=0-(-1)=1.
  \]
\end{example}

Resulta un poco trabajoso escribir todo ese texto sobre la antiderivada, para eso viene bien la notación siguiente:
\[
\big[G(x)\big]_a^b = G(b)-G(a).
\]
Así, el cálculo de las integrales de los ejemplos anteriores se puede escribir en una sola línea, dejando en claro los pasos, de la siguiente manera:
\begin{align*}
  \int_1^4 x^2 \dx &= \Big[ \frac{x^3}{3}\Big]_1^4 =\frac{4^3}3-\frac{1^3}3 = \frac{64}3-\frac13=21,
  \\
  \int_0^{\pi/2} \sen x\dx &= \Big[ -\cos x \Big]_{0}^{\pi/2} = (-\cos \frac\pi2)-(-\cos 0)=0-(-1)=1.
\end{align*}

En general, el cálculo de antiderivadas no es tan sencillo como el de derivadas, pero con lo que sabemos, y la observación que sigue, podemos calcular antiderivadas de una gran familia de funciones:

\begin{remark}
  Si $F$ y $G$ son antiderivadas de $f$ y $g$ respectivamente, entonces una antiderivada de $c\, f(x)+d\, g(x)$ es 
  $c\, F(x) + d\, G(x)$, cualesquiera sean las constantes $c$ y $d$.
  En efecto, como la derivada es lineal
  $$ 
  \dd[]{}{x}\big[c\, F(x) + d\, G(x) \big] = c\, F'(x)+d\,G'(x) = c\, f(x)+d\, g(x).
  $$
\end{remark}

Esto implica que la integral es lineal.

\begin{corollary}[Linealidad de la integral]
Si $f$ y $g$ son funciones continuas en $[a,b]$ y $c$ y $d$ son dos constantes, entonces
$$ 
\int_a^b   c\, f(x)+d\, g(x) \dx 
=
c\, \int_a^b  f(x) \dx 
+
d\, \int_a^b   g(x) \dx .
$$
\end{corollary}

\begin{proof}
  Sean $F$ y $G$ antiderivadas de $f$ y $g$, respectivamente.
  Por el comentario previo, $c\, F(x) + d\, G(x)$ es una antiderivada de $c\, f(x)+d\, g(x)$.
  Luego
  \begin{align*}
    \int_a^b   c\, f(x)+d\, g(x) \dx  &= \Big[c\, F(x) + d\, G(x)\Big]_a^b
    \\
    &= \big[c\, F(b) + d\, G(b)\big] - \big[c\, F(a) + d\, G(a)\big]
    \\
    &= c \big[F(b)-F(a)\big] + d \big[G(b)-G(a)\big] 
    \\
    &= c\, \int_a^b  f(x) \dx 
    +
    d\, \int_a^b   g(x) \dx .
    \qedhere
  \end{align*}
\end{proof}

\begin{example}
  Con este último resultado podemos calcular la integral de funciones un poco más complicadas:
  \begin{align*}
    \int_0^\pi 9x^2 - 2 \sen x\dx
    &= 9 \int_0^\pi x^2 \dx - 2 \int_0^\pi \sen x\dx
    \\
    &= 9 \Big[ \frac{x^3}3\Big]_0^\pi  - 2 \Big[-\cos x\Big]_0^\pi
    \\
    &= 9 \Big[ \frac{\pi^3}3-\frac{0^3}3\Big]  - 2 \Big[(-\cos \pi)-(-\cos 0)\Big]
    \\
    &= 9 \frac{\pi^3}3 - 2 \Big[(1)-(-1)\Big]
    = 3 \pi^3 - 4.
  \end{align*}
\end{example}

\begin{remark}
  !`Atención! Sólo \emph{las constantes} pueden sacarse fuera de la integral:
  \[
  \underbrace{\int_0^\pi x\, \sen x\dx}_{\text{número}} \neq \underbrace{x \  \overbrace{\int_0^\pi \sen x\dx}^{\text{número}}}_{\text{expresión variable}}.
  \]
  Para ver fácilmente que esto no es cierto, basta con observar que el lado izquierdo es un número y el lado derecho es una expresión variable, que aún depende de $x$.
\end{remark}

\subsubsection*{Ejercicios de la sección~\getcurrentref{chapter}.\getcurrentref{section}}

\begin{enumerate}
  \item Calcular las siguientes integrales
\begin{multicols}{2}
  \begin{enumerate}
    \item $\D \int_0^\pi \cos x \dx$.
    \item $\D \int_1^4 3 x^2 \dx$.
    \item $\D \int_1^e \frac1x \dx$.
    \item $\D \int_0^2 \big( x^2+3x+2\big) \dx$.
    \item $\D \int_0^1 \cosh x \dx$.
    \item $\D \int_1^4 \frac{1}{2\sqrt{x}} \dx$.
    \item $\D \int_4^9 \frac{1}{\sqrt x}\dx$.
    \item $\D \int_1^2 x^a \dx$, ($a\in \R$, $a\neq -1$).
  \end{enumerate}
\end{multicols}
\item Hallar el área de la región comprendida entre la gráfica de $f(x)=\sen x$ y el eje de abscisas, para $x$ en el intervalo dado:
\begin{multicols}{2}
  \begin{enumerate}
    \item $\D f(x)=4x-x^2$; $[0,4]$.
    \item $\D f(x)=x\sqrt{x}+1$; $[1,9]$.
    \item $\D f(x)=2\cos x$; $[-\pi/2,\pi/4]$.
    \item $\D f(x)=\sen x$; $[0,\pi]$.
  \end{enumerate}
\end{multicols}

\item Hallar el área de la región comprendida entre la gráfica de $f(x)=\sen x$ y el eje de abscisas, para $x$ entre $0$ y $2\pi$.

\item Calcular las siguientes integrales
\begin{multicols}{2}
  \begin{enumerate}
    \item $\D \int_2^5 (x-3) \dx$.
    \item $\D \int_2^5 |x-3| \dx$.
    \item $\D \int_{-4}^2 (2x+3) \dx$.
    \item $\D \int_{-4}^2 |2x+3| \dx$.
  \end{enumerate}
\end{multicols}

\item Calcular la derivada de las siguientes funciones:
\begin{multicols}{2}
  \begin{enumerate}
    \item $\D \int_1^x (t+2)^2 \dt$.
    \item $\D \int_1^x (\cos t-\sen t)^2 \dt$.
    \item $\D \int_0^{2x+1} \frac{1}{1+t^2} \dt$.
    \item $\D \int_x^5 e^{-t^2} \dt$.
  \end{enumerate}
\end{multicols}

\item Calcular las siguientes integrales:
  \begin{enumerate}
    \item $\D \int_0^4 f(x)\dx$, con 
    $\D f(x)=\begin{cases}
      2x+1, \quad \text{si $0\le x \le 1$},
      \\
      4-x, \quad \text{si $1<x\le 4$}.
    \end{cases}
      $
      \item $\D \int_{-2}^4 f(x)\dx$, con 
          $\D f(x)=\begin{cases}
        x^2, \quad \text{si $-2\le x < 0$},
        \\
        \frac12 x+2, \quad \text{si $0\le x\le 4$}.
          \end{cases}
        $
  \end{enumerate}

\item Calcular la integral $\D\int_0^{5} \dd[]{(e^{-x^2})}{x}\dx$.

\item Si $f$ es una función diferenciable con $f'$ continua en $[a,b]$, ?`a qué es igual $\D\int_a^b f'(x) \dx$?
\end{enumerate}



\section{Algunos problemas de áreas}

Al comienzo de este capítulo vimos que si $f$ es no negativa y continua en $[a,b]$, entonces $\int_a^b f(x)\dx$ da el área de la región delimitada por la gráfica de  $f$, y el eje $x$ entre las rectas $x=a$ y $x=b$.

\noindent
\begin{minipage}{.7\textwidth}
$$ \area(\Omega)=\int_a^b f(x)\dx.$$
\end{minipage}
\begin{minipage}{.29\textwidth}
  \centering
  \includegraphics[width=.9\textwidth]{pics/area-bajo-curva.png}
\end{minipage}

\begin{example}
  Hallar el área debajo de la gráfica de la función raíz cuadrada entre $x=0$ y $x=1$.

  Como $\sqrt{x}\ge0$ para todo $x$, dicha área se calcula de la siguiente manera:
  \[
  \int_0^1 \sqrt{x}\dx = \int_0^1 x^{1/2}\dx 
  = \Big[ \frac{x^{1/2+1}}{1/2+1} \Big]_0^1
  = \Big[ \frac23x^{3/2} \Big]_0^1
  = \frac23 \cdot 1 - \frac23 \cdot 0 = \frac23.
  \]
\end{example}

\centerline{\includegraphics[width=.9\textwidth]{pics/areas-1.png}}

\begin{example}
  Hallar el área de la región comprendida entre la curva $y=4-x^2$ y el eje $x$.

  En este caso no nos dicen entre qué valores hay que tomar $x$. Entonces, tenemos que ver las intersecciones entre la curva $y=4-x^2$ y el eje $x$. Las intersecciones se encuentran en $x=-2$ y $x=2$.
  Luego calculamos el área integrando entre $-2$ y $2$:
  \[
  \int_{-2}^2 (4-x^2)\dx = \Big[ 4\, x-\frac13 x^3 \Big]_{-2}^2
  = \frac{32}3.
  \]
\end{example}
\subsubsection*{Ejercicios de la sección~\getcurrentref{chapter}.\getcurrentref{section}}

\begin{enumerate}
  \item Graficar la región limitada por las curvas y calcular su área.
\begin{multicols}{2}
  \begin{enumerate}
    \item $\D y= \sqrt x$,\quad $\D y= x^2$.
    \item $\D y= 8-x^2$,\quad $\D y= x^2 $.
    \item $\D y= 5-x^2$,\quad $\D y= 3-x $.
    \item $\D y^2=2x$,\quad $\D x-y=4  $.
    \item $\D x-y^2+3=0 $,\quad $\D x-2y=0  $.
  \end{enumerate}
  
\end{multicols}

\item Para cada dibujo, escribir el área de la región sombreada como una suma de integrales.

\begin{center}
  \includegraphics[height=.22\textwidth]{pics/areas-7a.png}
  \hfil
  \includegraphics[height=.22\textwidth]{pics/areas-7b.png}

  \includegraphics[height=.28\textwidth]{pics/areas-7c.png}
\end{center}

\end{enumerate}  


\section{Integrales indefinidas}


\subsubsection*{Ejercicios de la sección~\getcurrentref{chapter}.\getcurrentref{section}}

\begin{enumerate}
  \item Calcular las siguientes integrales indefinidas:
\begin{multicols}{2}
  \begin{enumerate}
    \item $\D \int \frac{1}{x^4} \dx$
    \item $\D \int \big(ax+b\big) \dx$
    \item $\D \int \frac{1}{\sqrt{1+x}} \dx$
    \item $\D \int \bigg(\frac{x^3-1}{x^2}\bigg) \dx$
    \item $\D \int \bigg(\sqrt{x}-\frac{1}{\sqrt{x}}\bigg) \dx$
    \item $\D \int \big(\sen x+\cos x\big) \dx$
    \item $\D \int \big(\senh x+\cosh x\big) \dx$
    \item $\D \int \frac{g'(x)}{g(x)^2} \dx$
  \end{enumerate}
\end{multicols}

\item Hallar $f$ a partir de la información dada:
\begin{multicols}{2}
  \begin{enumerate}
    \item $f'(x)=2x-1$,\ \ $f(3)= $
    \item $f'(x)=ax+b$,\ \ $f(0)=c $
    \item $f'(x)=\sen x$,\ \ $f(0)= 2$
    \item $f'(x)=\cos x$,\ \ $f(\pi )=3 $
  \end{enumerate}
\end{multicols}

\item Un objeto se mueve a lo largo de un eje de coordenadas con una velocidad de $v(t)=6t^2-6$ metros por segundo. Su posición inicial (posición en tiempo $t=0$) es 2 unidades a la izquierda del origen.
\begin{enumerate}
  \item Hallar la posición del objeto $3$ segundos más tarde.
  \item Representar esquemáticamente el movimiento del objeto como en el ejemplo~\ref{ej:velocidad}.
  \item ?`Cuál es la distancia entre la posición del objeto a tiempo $t=0$ y la posición a tiempo $t=3$?
  \item ?`Cuál es la distancia total recorrida por el objeto entre tiempo $t=0$ y $t=1$?
\end{enumerate}
\end{enumerate}    


\section{Integración por sustitución}


\subsubsection*{Ejercicios de la sección~\getcurrentref{chapter}.\getcurrentref{section}}

\begin{enumerate}
  \item Calcular las siguientes integrales indefinidas mediante el método de sustitución:

\begin{multicols}{2}
  \begin{enumerate}
    \item $\D \int \frac{1}{(2-3x)^2} \dx$.
    \item $\D \int \sqrt{ax+b} \dr$.
    \item $\D \int \frac{t}{(4t^2+9)^2} \dt$.
    \item $\D \int x^2 (1+x^3)^{1/4} \dx$.
    \item $\D \int \frac{b^3 \, x^3}{\sqrt{1-a^4x^4}} \dx$.
    \item $\D \int \frac{1}{4x+6}\dx$.
    \item $\D \int \frac{x^3}{1+x^4}\dx$.
    \item $\D \int e^{ax+b}\dx$.
    \item $\D \int x^3 e^{x^4+1}\dx$.
  \end{enumerate}
\end{multicols}

\item Calcular las siguientes integrales definidas mediante el método de sustitución:

\begin{multicols}{2}
  \begin{enumerate}
    \item $\D \int_{0}^{1} x(x^2+1)^3 \dx$.
    \item $\D \int_{-1}^{1} \frac{r}{(1+r^2)^4} \dr$.
    \item $\D \int_{0}^{3} \frac{r}{\sqrt{r^2+16}}  \dr$.
    \item $\D \int_{0}^{a} y\sqrt{a^2-y^2} \dy$.
  \end{enumerate}
\end{multicols}

\item Calcular las siguientes integrales indefinidas utilizando el cambio de variables indicado:

\begin{multicols}{2}
  \begin{enumerate}
    \item $\D \int x \sqrt{x+1} \dx $, $u = x+1$.
    \item $\D \int x \sqrt{2x-1} \dx $, $u = 2x-1$.
    \item $\D \int t\,(2t+3)^8 \dt $, $u = 2t+3$.
    \item $\D \int \frac{x+3}{\sqrt{x+1}}\dx $, $u = x+1$.
  \end{enumerate}
\end{multicols}
\end{enumerate}  

\section{Integración por partes}


\subsubsection*{Ejercicios de la sección~\getcurrentref{chapter}.\getcurrentref{section}}

\begin{enumerate}
  \item Calcular las siguientes integrales:
\begin{multicols}{2}
  \begin{enumerate}
    \item $\D \int x\,e^{-x} \dx$.
    \item $\D \int_0^2 x \, 2^x  \dx$.
    \item $\D \int x^2\, e^{-x^3} \dx$.
    \item $\D \int x \, \ln x^2 \dx$.
    \item $\D \int_0^1 x^2\, e^{-x} \dx$.
    \item $\D \int \frac1{x\, (\ln x)^3} \dx$.
    % \item $\D \int  \dx$.
  \end{enumerate}
  
\end{multicols}
\item Hallar el área de la región limitada por la gráfica de $f$ y el eje $x$:
\begin{multicols}{2}
  \begin{enumerate}
    \item $f(x)=\arcsen x$, $x\in[0,1/2]$.
    \item $f(x)=x\,e^{-2x}$,  $x\in[0,2]$.
  \end{enumerate}
\end{multicols}
\end{enumerate}  



\section{Integrales impropias}


\subsubsection*{Ejercicios de la sección~\getcurrentref{chapter}.\getcurrentref{section}}

\begin{enumerate}
  \item Supongamos que $f$ y $g$ son continuas, $a<b$ y 
\[
\int_a^b f(x)\dx > \int_a^b g(x)\dx.
\]
\begin{enumerate}
  \item ?`Se deduce necesariamente que $\int_a^b \big[ f(x)-g(x)\big]\dx > 0$?
  \item ?`Se deduce necesariamente que $f(x)>g(x)$, para todo $x\in [a,b]$?
  \item ?`Se deduce necesariamente que existe $x\in[a,b]$ tal que $f(x)>g(x)$?
  \item ?`Se deduce necesariamente que $\Big|\int_a^b f(x)\dx\Big| > \Big|\int_a^b g(x)\dx\Big| $?
\end{enumerate}

\item Supongamos que $f$ es continua, $a<b$ y 
\[
\int_a^b f(x)\dx =0.
\]
\begin{enumerate}
  \item ?`Se deduce necesariamente que $f(x)=0$, para todo $x\in [a,b]$?
  \item ?`Se deduce necesariamente que existe $x\in[a,b]$ tal que $f(x)=0$?
  \item ?`Se deduce necesariamente que $\Big|\int_a^b f(x)\dx\Big| =0 $?
  \item ?`Se deduce necesariamente que $\int_a^b |f(x)|\dx =0 $?
\end{enumerate}
\end{enumerate}


\section{Algunas propiedades adicionales de la integral}


\subsubsection*{Ejercicios de la sección~\getcurrentref{chapter}.\getcurrentref{section}}

\begin{enumerate}
  \item La base de un sólido es el círculo $x^2 + y^2 = 7$. Hallar el volumen de dicho sólido si las secciones transversales perpendiculares al eje $x$ son
\begin{multicols}{2}
  \begin{enumerate}
    \item Cuadrados.
    \item Triángulos equiláteros.
  \end{enumerate}
\end{multicols} 

\item La base de un sólido es la región limitada por la elipse $4x^2 + 9y^2 = 36$. Hallar el volumen del sólido sabiendo que las secciones transversales perpendiculares al eje $x$ son
\begin{multicols}{2}
  \begin{enumerate}
    \item Triángulos equiláteros.
    \item Cuadrados.
  \end{enumerate}
\end{multicols} 

\item La base de un sólido es la región limitada por $y = x^2$ e $y = 4$. Hallar el volumen sabiendo que las secciones perpendiculares al eje $x$ son:
\begin{multicols}{2}
  \begin{enumerate}
    \item Cuadrados.
    \item Semicírculos.
    \item Triángulos equiláteros.
  \end{enumerate}
\end{multicols} 

\item La base de un sólido es la región comprendida entre las parábolas $x = y^2$ y $x = 3 - 2y^2$. Hallar el volumen del sólido sabiendo que las secciones perpendiculares al eje $x$ son 
\begin{multicols}{2}
  \begin{enumerate}
    \item Rectángulos de altura $h$.
    \item Triángulos equiláteros.
    \item Triángulos rectángulos isósceles, cada uno con la hipotenusa sobre el plano $xy$
  \end{enumerate}
\end{multicols} 

\item Hallar la fórmula para el volumen de un cono truncado en
función de su altura $h$, el radio de la base inferior $R$ y el radio de
la base superior $r$.

\centerline{\includegraphics[width=.3\textwidth]{pics/cono-truncado.png}}

\item Un plano corta una esfera de radio $r$ a una altura de $h$ unidades
por encima del ecuador ($0 < h < r$). La parte superior de la
esfera se llama casquete. Deducir la fórmula de su volumen.

\item  La región que se muestra en la figura se rota alrededor de eje $y$
para formar un recipiente de forma parabólica. La unidades
indicadas están en metros. ?`Cuál es el volumen del recipiente?

\centerline{\includegraphics[width=.3\textwidth]{pics/recipiente-parabolico.png}}


\end{enumerate}


\section{El método de integración por fracciones simples}


\subsubsection*{Ejercicios de la sección~\getcurrentref{chapter}.\getcurrentref{section}}

\begin{enumerate}
\item 
\end{enumerate}



\subsection*{Ejercicios del capítulo~\getcurrentref{chapter}}



