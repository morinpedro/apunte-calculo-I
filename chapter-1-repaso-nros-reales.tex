\chapter{Repaso de Números Reales}
\label{Cap:Reales}

\section{Propiedades de cuerpo ordenado}

Recordamos a continuación los axiomas de cuerpo ordenado que cumple el conjunto $\R$ de los números reales.

Las propiedades básicas de la suma son cuatro:

\begin{description}
    \item[$S_1$. Conmutatividad de la suma:] Cualesquiera sean los números reales $a$ y $b$ vale
    \[ a+b = b+a.\]

    \item[$S_2$. Asociatividad de la suma:] Cualesquiera sean los números reales $a$, $b$ y $c$, vale
    \[
    a + (b+c) = (a+b) + c.
    \]

    \item[$S_3$. Existencia de elemento neutro para la suma:] Existe un número real llamado \emph{cero} y que indicamos 0 tal que, para todo número real $a$ vale:
    \[ a+0 = a.\]

    \item[$S_4$. Existencia del inverso aditivo:] Dado un número real $a$, existe un número real que llamamos \emph{inverso aditivo de $a$} o también \emph{opuesto de $a$}, e indicamos $-a$ tal que
    \[ a+(-a) = 0.\]

\end{description}

Las propiedades básicas del producto son también cuatro, y se corresponden exactamente con las de la suma:

\begin{description}
    \item[$P_1$. Conmutatividad del producto:] Cualesquiera sean los números reales $a$ y $b$ vale
    \[ a\cdot b = b\cdot a.\]

    \item[$P_2$. Asociatividad del producto:] Cualesquiera sean los números reales $a$, $b$ y $c$, vale
    \[
    a \cdot (b \cdot c) = (a \cdot b) \cdot c.
    \]

    \item[$P_3$. Existencia de elemento neutro para el producto:] Existe un número real distinto de cero, llamado \emph{uno} y que indicamos 1 tal que, para todo número real $a$ vale:
    \[ a \cdot 1 = a.\]

    \item[$P_4$. Existencia del inverso multiplicativo:] Dado un número real $a$, distinto de cero, existe un número real que llamamos \emph{inverso multiplicativo de $a$} o también \emph{recíproco de $a$}, e indicamos $a\inv$ tal que
    \[ a \cdot a\inv = 1.\]

\end{description}

También tenemos una propiedad que vincula la suma con el producto:

\begin{description}
    \item[$D$. Propiedad distributiva:] Cualesquiera sean los números reales $a$, $b$ y $c$ vale:
    \[ a\cdot(b+c) = a\cdot b+a\cdot c.\]
\end{description}

Finalmente, existe en los números reales un \emph{orden}. La notación $a<b$ significa
\emph{$a$ es menor que $b$} y es exactamente lo mismo que $b>a$, que se lee \emph{$b$ es mayor que $a$}. Para el orden, tenemos las siguientes propiedades:

\begin{description}
    \item[$O_1$. Tricotomía:] Si $a$ y $b$ son números reales, vale una y sólo una de las siguientes posibilidades:
    \[
    a<b,\qquad a=b, \qquad a>b.
    \]

    \item[$O_2$. Transitividad:] Si $a$, $b$ y $c$ son números reales que verifican
    \[ a<b \qquad\text{y}\qquad b<c\]
    entonces necesariamente \ $a<c$.

    \item[$O_3$. Monotonía de la suma:] Si $a$ y $b$ son números reales que verifican
    \[ a<b, \]
    entonces cualquiera sea el número real $c$ vale:
    \[ a+c < b+c.\]

    \item[$O_4$. Monotonía del producto:] Si $a$ y $b$ son números reales que verifican
    \[ a<b, \]
    entonces cualquiera sea el número real $c$ mayor que cero vale:
    \[ a \cdot c < b \cdot c.\]


\end{description}

\mauri{Se puede probar como ejercicio que 
$a < b$ si y solo si $-b < -a$, con lo cual, 
por la propiedad $O_4$ anterior tenemos que:
Si $a$ y $b$ son números reales que verifican
    \[ a<b, \]
    entonces cualquiera sea el número real $c$ menor que cero vale:
    \[ a \cdot c > b \cdot c.\]}

Recordamos algunas propiedades que se deducen de estos axiomas:

\begin{description}
    \item[Propiedad cancelativa para la suma:] Si $a$, $b$ y $c$ son números reales tales que
    \[ a+c = b+c\]
    entonces \ $a=b$.

    \item[Unicidad del cero:] Si $a$ y $b$ son números reales tales que
    \[a+b=a,\]
    entonces debe ser \ $b=0$.

    \item[$a\cdot 0 = 0$:] Cualquiera sea el número real $a$, vale que $a\cdot 0 = 0$.

    \item[Propiedad cancelativa para el producto:] Si $a$, $b$ y $c$ son números reales tales que $c\neq 0$ y
    \[ a\cdot c = b \cdot c\]
    entonces \ $a=b$.

    \item[Reglas de los signos:]
    \begin{itemize}
        \item Si $a$ es un número real, entonces \ $-(-a) = a$.
        \item Si $a$ y $b$ son números reales, entonces \ $(-a)\cdot b = -(a\cdot b)$.
        \item Si $a$ y $b$ son números reales, entonces \ $(-a)\cdot (-b) = a\cdot b$.
    \end{itemize}

    \item[Por último: $0<1$.]
\end{description}

Finalmente, decimos que $a$ es menor o igual a $b$, y escribimos $a\le b$, cuando $a<b$ o $a=b$.

\subsubsection*{Ejercicios de la sección~\getcurrentref{chapter}.\getcurrentref{section}}

\begin{enumerate}
    
    \item Probar que si $a,b,x,y \in \R$, $a<b$ y $x<y$, entonces $a+x < b+y$.

    \item Probar que si $a,b\in\R$ y $a<b$, entonces $\D a<\frac{a+b}2<b$.

    \item Verificar que:
    \begin{enumerate}
        \item $n < n + 5 < 2 n$, para todo $n\in\N$, $n > 5$.
        \item $n < n + 5 < 2 n$, para todo $n\in\N$, $n > 50$.
        \item $\frac n2 < n - 10 < n$, para todo $n\in\N$, $n > 20$.
        \item $\frac n2 < n - 10 < n$, para todo $n\in\N$, $n > 100$.
        \item $5\, n^2 < 5\,n^2 + n - 5 < 6\, n^2$, para todo $n\in\N$, $n > 5$.
        \item $4\, n^2 < 5\,n^2 - n  < 5\, n^2$, para todo $n\in\N$, $n > 5$.
    \end{enumerate}

    \item Hallar $N_0\in\N$ tal que:
    \begin{enumerate}
        \item $n < n + 30 < 2 n$, para todo $n\in\N$, $n > N_0$.
        \item $\frac n2 < n - 30 < n$, para todo $n\in\N$, $n > N_0$.
        \item $5\, n^2 < 5\,n^2 + n - 5 < 6.1\, n^2$, para todo $n\in\N$, $n > N_0$.
        \item $4.9\, n^2 < 5\, n^2 - n  < 5\, n^2$, para todo $n\in\N$, $n > N_0$.
        \item $n^2 < 2\,n^2 - 10\, n + 8 < 3\, n^2$, para todo $n\in\N$, $n > N_0$.
        \item $2\, n^3 < 3\,n^3 - 5 n^2 + 30 < 3\, n^3$, para todo $n\in\N$, $n > N_0$.
    \end{enumerate}


\end{enumerate}

\section{Cotas superiores e inferiores, máximo y mínimo}

\begin{definition}
    Dado un conjunto $A$ de números reales, se dice que $c$ es una cota inferior de $A$ si
    \[
    c \le a,\qquad \forall a\in A.
    \]
    Es decir, un número real es una cota inferior de un conjunto cuando es menor o igual que todos los elementos del conjunto.
\end{definition}

\begin{itemize}
\item Existen conjuntos que no tienen cotas inferiores. Pensar ejemplos.
\item Si $c$ es una cota inferior, entonces todos los elementos de $(-\infty,c]$ son cotas inferiores. Demostrar
\item Si un conjunto tiene una cota inferior, se dice que es \emph{acotado inferiormente}.
\end{itemize}

\begin{definition}
Dado un conjunto $A$ de números reales, se dice que $c$ es una cota superior de $A$ si
\[
a \le c,\qquad \forall a\in A.
\]
Es decir, un número real es una cota superior de un conjunto cuando es mayor o igual que todos los elementos del conjunto.
\end{definition}

\begin{itemize}
    \item Existen conjuntos que no tienen cotas superiores. Pensar ejemplos.
    \item Si $c$ es una cota superior, entonces todos los elementos de $[c,\infty)$ son cotas superiores. Demostrar
    \item Si un conjunto tiene una cota superior, se dice que es \emph{acotado superiormente}.
\end{itemize}

Cuando un conjunto es acotado inferiormente y acotado superiormente, se dice simplemente que es \emph{acotado}.

\begin{definition}
    Dado un conjunto $A$ de números reales, se dice que $m$ es un mínimo de $A$ si
    \[ m\in A,\qquad \text{y} \qquad m\text{ es una cota inferior de $A$}, \]
    o, lo que es lo mismo,
    \[ m\in A,\qquad \text{y} \qquad m\le a,\ \forall a\in A. \]
\end{definition}

\begin{itemize}
    \item Si un conjunto no es acotado inferiormente, entonces no tiene mínimo.
    \item Hay conjuntos acotados inferiormente que no tienen mínimo. Pensar ejemplos.
    \item Hay conjuntos acotados inferiormente que sí tienen mínimo. Pensar ejemplos.
    \item Un conjunto finito siempre tiene mínimo.
    \item Si $m$ y $m'$ son mínimos de un conjunto $A$, entonces $m=m'$. Es decir, el mínimo es único.
    Como el mínimo es único, cuando un conjunto $A$ tiene mínimo, se denota con $\min A$.

\end{itemize}


\begin{definition}
    Dado un conjunto $A$ de números reales, se dice que $M$ es un máximo de $A$ si
    \[ M\in A,\qquad \text{y} \qquad M\text{ es una cota superior de $A$}, \]k
    o, lo que es lo mismo,
    \[ M\in A,\qquad \text{y} \qquad a\le M,\ \forall a\in A. \]
\end{definition}

\begin{itemize}
    \item Si un conjunto no es acotado superiormente, entonces no tiene máximo.
    \item Hay conjuntos acotados superiormente que no tienen máximo. Pensar ejemplos.
    \item Hay conjuntos acotados superiormente que sí tienen máximo. Pensar ejemplos.
    \item Un conjunto finito siempre tiene máximo.
    \item Si $M$ y $M'$ son máximos de un conjunto $A$, entonces $M=M'$. Es decir, el máximo es único.
    Como el máximo es único, cuando un conjunto $A$ tiene máximo, se denota con $\max A$.
\end{itemize}

\subsubsection*{Ejercicios de la sección~\getcurrentref{chapter}.\getcurrentref{section}}

\begin{enumerate}

    \item Probar que el intervalo $(0,1)$ es acotado inferiormente pero no tiene mínimo.

    \item Probar que el intervalo $(0,1)$ es acotado superiormente pero no tiene máximo.

    \item Sea $A$ un conjunto de números reales con mínimo $m$. Sea $A'$ el conjunto definido por $A'=\{-x : x \in A\}$. Probar que $\max A' = -\min A$.

\end{enumerate}

\section{Supremo, ínfimo, axioma de completitud}

\begin{definition}
    Dado un conjunto $A$ de números reales, se dice que $c$ es un supremo de $A$ si cumple las dos siguientes propiedades:
    \begin{enumerate}[{\rm(i)}]
        \item $c$ es una cota superior de $A$;
        \item Si $d$ es cota superior de $A$, entonces $c \le d$.
    \end{enumerate}
    Es decir, $c$ es un supremo de $A$ si es \emph{la menor de las cotas superiores}.

\end{definition}


\begin{itemize}
    \item Si un conjunto no es acotado superiormente, entonces no tiene supremo.
    \item Si un conjunto tiene máximo, entonces el máximo es el supremo.
    \item Hay conjuntos acotados superiormente que tienen supremo, pero no tienen máximo. Pensar ejemplos.
    \item Si $c$ y $c'$ son supremos de un conjunto $A$, entonces $c=c'$. Es decir, el supremo es único.
    Como el supremo es único, cuando un conjunto $A$ tiene supremo, se denota con $\sup A$.
    \item Si un conjunto tiene supremo, el supremo puede pertenecer al conjunto o no. Si pertenece al conjunto, entonces el conjunto tiene máximo y el máximo y el supremo coinciden. Si el supremo no pertenece al conjunto, entonces el conjunto no tiene máximo.
\end{itemize}



En este curso tomaremos la \emph{propiedad de completitud} como un axioma, que se agrega a los vistos anteriormente. En cursos de cálculo avanzado se hace una construcción de los números reales a partir de los racionales y se demuestra que esta construcción implica la validez de la propiedad de completitud, que se enuncia como sigue:

\paragraph{Propiedad de completitud.} Si $A$ es un conjunto no vacío de números reales, y $A$ es acotado superiormente, entonces existe el supremo de $A$.


\begin{definition}
    Dado un conjunto $A$ de números reales, se dice que $c$ es un ínfimo de $A$ si cumple las dos siguientes propiedades:
    \begin{enumerate}[{\rm(i)}] 
        \item $c$ es una cota inferior de $A$;
        \item Si $d$ es cota inferior de $A$, entonces $c \ge d$.
    \end{enumerate}
    Es decir, $c$ es un ínfimo de $A$ si es \emph{la mayor de las cotas inferiores}.

\end{definition}


\begin{itemize}
    \item Si un conjunto no es acotado inferiormente, entonces no tiene ínfimo.
    \item Si un conjunto tiene mínimo, entonces el mínimo es el ínfimo.
    \item Hay conjuntos acotados inferiormente que tienen ínfimo, pero no tienen mínimo. Pensar ejemplos.
    \item Si $c$ y $c'$ son ínfimos de un conjunto $A$, entonces $c=c'$. Es decir, el ínfimo es único.
    Como el ínfimo es único, cuando un conjunto $A$ tiene ínfimo, se denota con $\inf A$.
    \item Si un conjunto tiene ínfimo, el ínfimo puede pertenecer al conjunto o no. Si pertenece al conjunto, entonces el conjunto tiene mínimo y el mínimo y el ínfimo coinciden. Si el ínfimo no pertenece al conjunto, entonces el conjunto no tiene mínimo.
\end{itemize}


\paragraph{Caracterización del supremo.} Hay una forma de caracterizar al supremo de un conjunto que resulta de mucha utilidad:

\begin{proposition}\label{P:supremo-caracterizacion}
    Sea $A$ un conjunto acotado superiormente y no vacío (por lo que tiene supremo).
    Entonces el número real $c$ es el supremo de $A$ si y sólo si cumple las dos siguientes condiciones:
    \begin{enumerate}[{\rm(i)}]
        \item $c$ es cota superior de $A$;
        \item para cada $\epsilon >0$ existe $a\in A$ tal que $c-\epsilon < a$.
    \end{enumerate}
\end{proposition}

\begin{proof}
%    \completar
\mauri{Por la definición de supremo, la primera condición es la misma, por lo tanto es trivial.}

\mauri{Supongamos primero que $c=\sup A$ y sea
$\epsilon >0$. Es claro que $c-\epsilon < c$, 
entonces por la definición de supremo 
$c-\epsilon$ no puede ser cota superior de $A$, 
es decir, no puede ocurrir que 
$a\leq c-\epsilon$, para todos los elementos de $A$. Entonces, existe $a\in A$ tal que $c-\epsilon < a$.}

\mauri{Por otro lado, si $d$ es una cota superior 
de $A$, debemos ver que $d \geq c$. Esto tiene que 
pasar, puesto que si no ocurre, se tiene que 
$d < c$ y considerando $\epsilon = c-d > 0$, la 
hipótesis afirma que existe $a \in A$ tal que 
\[ c - (c - d) = d < a\,,  \]
lo que contradice la definición de cota superior.}

\end{proof}


De manera análoga se puede caracterizar el ínfimo.

\begin{proposition}\label{P:infimo-caracterizacion}
    Sea $A$ un conjunto acotado inferiormente y no vacío (por lo que tiene ínfimo).
    Entonces el número real $c$ es el ínfimo de $A$ si y sólo si cumple las dos siguientes condiciones:
    \begin{enumerate}[{\rm(i)}]
        \item $c$ es cota inferior de $A$;
        \item para cada $\epsilon >0$ existe $a\in A$ tal que $c+\epsilon > a$.
    \end{enumerate}
\end{proposition}

\begin{proof}
    Ejercicio.
\end{proof}

\paragraph{Principio de Arquímedes (o propiedad de arquimedianidad).} Como consecuencia de la propiedad de completitud podemos enunciar y probar el principio de Arquímedes.

\begin{proposition}
    Si $a$ es un número real cualquiera, entonces existe un número natural $n$ tal que $n>a$.
\end{proposition}

\begin{proof}
    Supongamos, por el contrario, que existe un número real $a$ tal que $n \le a$, para todo $n \in \N$.
    Luego $\N$ es un conjunto acotado superiormente, y no vacío, pues $1\in \N$. Por la propiedad de completitud, el conjunto de los naturales tiene un supremo, que llamamos $c$, es decir $c = \sup \N$.

    Luego, por la proposición~\ref{P:supremo-caracterizacion}, para $\epsilon = 1$, existe $n_0 \in \N$ tal que $n_0 > c-\epsilon = c-1$. Luego $n_0 + 1 > c$, pero como $\N$ es inductivo, $n_0+1\in\N$. Esto contradice que $c$ sea una cota superior de $\N$. Esta contradicción provino de suponer que existe un número real $a$ tal que $n \le a$, para todo $n \in \N$.
    Luego, si $a\in\R$, existe necesariamente un número natural $n>a$.
\end{proof}

\begin{corollary}\label{1sobrenmenorqueeps}
    Si $\epsilon$ es un número real positivo, entonces existe un número natural $n$ tal que $\D\frac1n < \epsilon$.
\end{corollary}

\begin{proof}
    Ejercicio.
\end{proof}

\begin{lemma}\label{P:densidad de Q en R}
    Entre dos número reales $a$ y $b$ 
    existe un número racional $t$. 
    Más precisamente, si $a < b$ existe 
    $t \in \Q$ tal que $a < t < b$.
\end{lemma}

\begin{proof}
    Para demostrar este Lema haremos uso 
    de los resultados que venimos obteniendo, considerando una serie de 
    casos.
    \begin{itemize}
        \item Si $a = 0$ y $b > 0$ entonces por el Colorario 
        \ref{1sobrenmenorqueeps}, existe $\niN$ tal que $0<\frac1n < b$ y tomando $t=\frac1n$ este caso está probado.
        \item Si $a < 0$ y $b =0$ observamos que $0 < -a$ y por el 
        item anterior existe $\frac1n$ tal que $0 < \frac1n < -a$. Definiendo $t=-\frac1n \in \Q$ se tiene el resultado.
        \item Si $a < 0 < b$ entonces tomamos $t=0$ y nada hay que probar.
        \item Si $0 < a < b$ vamos a suponer por un ratito que este caso lo probamos.
        \item Si $a < b < 0$ entonces se 
        cumple que $0 < -b < -a$ y por el item anterior existe $r\in \Q$ tal que $-b < r < -a$, o equivalentemente $a < t < b$, tomando $t=-r$.
    \end{itemize}
    En resumen, para demostrar el Lema será suficiente probar el caso en que ambos son positivos. Para esto, consideremos $\epsilon=b-a > 0$ y 
    $\niN$ tal que $\frac1n < b-a$ (tal $n$ existe por el Corolario \ref{1sobrenmenorqueeps}). 
    Consideremos el conjunto
    \[ A=\left\{p \in \N\,: ~ p\,\frac1n > a\right\}\,. \]
    Observar que $A$ contiene a todos los 
    naturales $p$ tales que $\frac{p}{n}$ supera a $a$, como estamos buscando uno que cumpla con esto pero que también sea menor que $b$ debemos ser cautelosos al definir. 
    Notemos además que $A \subset \N$ y $A \neq \emptyset$ por el principio de Arquímedes. Luego, por el principio de la buena ordenación de los naturales tiene sentido definir 
    \[  m = \min A \,.\]
    Probaremos que $\frac{m}{n}$ es el 
    racional buscado. En efecto, como $m \in A$, es claro que 
    \[ m \, \frac1n = \frac{m}{n} > a\,.\]
%
    Por otro lado, si suponemos que 
    $\frac{m}{n} \geq b$ entonces 
    \[  (m-1)\,\frac1n = \frac{m}{n} - \frac1n \geq b - (b - a) = a\,,\]
    con lo cual $m-1 \in A$ y como $m-1 < m$ contradice el hecho que $m$ sea el mínimo. En conclusión, se tiene que $a < \frac{m}{n} < b$ y la proposición queda probada.
\end{proof}

\subsubsection*{Ejercicios de la sección~\getcurrentref{chapter}.\getcurrentref{section}}

\begin{enumerate}

    \item Sea $A$ un subconjunto acotado de $\R$ y sea $B\subseteq A$ no vacío. Probar que: 
    \begin{enumerate}
        \item Toda cota superior de $A$ es cota superior de $B$.
        \item Toda cota inferior de $A$ es cota inferior de $B$.
    \end{enumerate}
    
    \item \mara{Sea $A$ un subconjunto acotado de $\R$ y sea $B\subseteq A$ no vacío. Probar que $\inf{A}\leq\inf{B}\leq\sup{B}\leq\sup{A}$.}

    \item* Enunciar y demostrar una proposición análoga a la Proposición~\ref{P:supremo-caracterizacion} para el ínfimo.

    \item \mara{Sea $a$ un número real tal que $0\leq a <\epsilon$ para todo $\epsilon>0$. 
        Probar que $a=0$. 
            
    Análogamente, demostrar que si $a$ y $b$ son dos números reales tales que $a<b+\epsilon$ para todo $\epsilon>0$, entonces $a\leq b$.}
\end{enumerate}

\section{Potencias de exponente real}

\subsection{Potencias naturales}

La definición rigurosa de potencia de un número real con exponente natural se hace por inducción:
Sea $a\in\R$:
\begin{itemize}
    \item $a^1 = a$
    \item si $n\in \N$, entonces $a^{n+1} = a^n \cdot a$
\end{itemize}
De esta forma, queda bien definido $a^n$, para todo $n \in \N$.

A partir de esta definición, podemos demostrar, también por inducción, la siguiente proposición:

\begin{proposition}\label{P:potencias naturales}
    Sea $a\in\R$ y sean $m,n\in \N$. Entonces
    \begin{enumerate}[{\rm (i)}]
        \item $a^{m+n} = a^m \cdot a^n$;
        \item $a^{m\cdot n} = (a^m)^n$.
    \end{enumerate}
\end{proposition}

\begin{proof}
    Ejercicio.
\end{proof}

La siguiente proposición tiene una interesante demostración por inducción de un hecho que no es \emph{obvio} como los de la proposición anterior.

\begin{proposition}[Desigualdad de Bernoulli]\label{P:Bernoulli}
    Si $h \in \R$ y $h > -1$, entonces, para todo $n \in \N$,
    \[
    (1+h)^n \ge 1 + n h.
    \]
\end{proposition}

\begin{proof}
    Ejercicio.
\end{proof}


\subsection{Potencias enteras}

La definición de potencia para números enteros se hace de tal manera que siga siendo cierta la Proposición~\ref{P:potencias naturales}.

Para definir $a^0$, observamos que para que sea cierta esa proposición, hace falta que, si $n\in\N$,
\[
a^0 \cdot a^n = a^{0+n} = a^n.
\]
Si $a\neq 0$, podemos multiplicar por $\big(a^n\big)^{-1}$,
y por lo tanto estamos obligados a definir definir $a^0 = 1$, para $a\neq 0$.

Definimos ahora $a^{-n}$ para $n\in \N$, y $a\neq 0$
. Nuevamente, por la proposición mencionada, hace falta que, si $n\in\N$,
\[
a^{-n} \cdot a^n = a^{-n+n} = a^0 = 1,
\]
y por lo tanto estamos obligados a definir definir $a^{-n} = \big( a^n \big)^{-1} = \frac{1}{a^n}$.

Con estas definiciones resulta que

\begin{proposition}\label{P:potencias enteras}
    Sea $a\in\R$ y sean $m,n\in \Z$. Entonces
    \begin{enumerate}[{\rm (i)}]
        \item $a^{m+n} = a^m \cdot a^n$;
        \item $a^{m\cdot n} = (a^m)^n$.
    \end{enumerate}
\end{proposition}

\begin{proof}
    Ejercicio.
\end{proof}

\subsection{Potencias racionales}
% Ver Noriega, pág. 48, columna 1.

Hasta ahora hemos logrado definir:
\begin{itemize}
    \item $a^n$, para todo número real $a$ y todo $n\in \N$.
    \item $a^n$, para todo número real $a\neq 0$ y todo $n \in \Z$. 
\end{itemize}

La definición de $a^n$ para $n\in\Z$ se basó en que se siga cumpliendo la igualdad $a^{m+n}=a^m\cdot a^n$. 

Ahora vamos a definir $a^{m/n}$, para $m\in\Z$, $n\in\N$ de manera que siga siendo válida la otra propiedad de la potenciación: $\big(a^m\big)^n = a^{m\cdot n}$.

Supongamos ahora que estuviese definido $a^{m/n}$ para $m\in\Z$, $n\in\N$. Entonces, suponiendo válida la propiedad mencionada en el párrafo anterior, debería ser
\[
    \big(a^{m/n}\big)^n = a^{\frac mn n} = a^m,
\]
y por lo tanto, si $a^{m/n}$ elevado a la $n$ da $a^m$, debe ser $a^{m/n}$ la raíz $n$-ésima de $a^m$: $\D a^{m/n} = \sqrt[n]{a^m}$. En base a este razonamiento se define.

\begin{definition}
    Sea $a\in\R$, $a>0$. Si $r \in \Q$, entonces 
    existen $m\in\Z$ y $n\in\N$ tales que \mauri{$r=m/n$}. Definimos
    \[
        \mauri{a^r} = a^{m/n} = \sqrt[n]{a^m}.
    \]
\end{definition}

Se habrá dado cuenta que todo racional puede escribirse como cociente de un entero por un natural, pero de múltiples formas, por ejemplo: $1/2 = 2/4 = 3/6$. Para que esta definición sea correcta, hace falta que para cualquiera de estas expresiones, la definición dé el mismo resultado. Para eso hace falta pedir que $a>0$.

Con esta nueva definición, se siguen cumpliendo las propiedades que hemos querido mantener:

\begin{proposition}
    Sea $a$ un número real positivo y sean $t,s\in\Q$. Entonces
    \begin{enumerate}[{\rm (i)}]
        \item $a^{t+s} = a^t \cdot a^s$;
        \item $a^{t\cdot s} = (a^t)^s$.
    \end{enumerate}
\end{proposition}

\begin{proof}
    Ejercicio.
\end{proof}

También se cumplen las siguientes propiedades:

\begin{proposition}\label{P:potencia-monotona}
    Sean $a,b$ números reales tales que $0<a<b$ y sea $t$ un número racional positivo. Entonces $a^t < b^t$.
\end{proposition}

\begin{proof}
    Ejercicio.
\end{proof}

\begin{proposition}\label{P:exponencial-monotona}
    Sean $a$ un número real y sean $t,s$ números racionales tales que $t<s$, entonces:
    \begin{enumerate}[{\rm (i)}]
        \item Si $a>1$, entonces $a^t < a^s$.
        \item Si $0<a<1$, entonces $a^t > a^s$.
    \end{enumerate}
\end{proposition}

\begin{proof}
    Ejercicio.
\end{proof}

\subsection{Potencias reales}

Según la Proposición~\ref{P:exponencial-monotona}, si $a>1$, la función exponencial dada por $f_a:\Q\to\R$, $f_a(t)= a^t$ es creciente. Es decir, $t<s$ implica que $a^t < a^s$. Si $a=1$, dicha función es constantemente 1, $f_a(t)=1$, para todo $t\in\Q$.
Finalmente, si $0<a<1$, la función $f_a$ es decreciente.

Queremos ahora dar sentido a la definición de $a^x$, para $a>0$ y $x\in\R$, de manera que se siga cumpliendo lo enunciado en el párrafo anterior.

Claramente, para $a=1$, definimos $1^x= 1$, para todo $x\in\R$.
Para los otros casos, necesitamos trabajar un poco más.

\begin{lemma}\label{L:exponencial-continuidad}
    Sea $a>1$ y sea $x\in\R$, arbitrario. 
    Entonces, dado $\epsilon>0$ existen $r,r'\in\Q$ tales que
    \begin{itemize}
        \item $r<x<r'$;
        \item $0<a^{r'} - a^{r} < \epsilon$.
    \end{itemize}
\end{lemma}

\begin{proof}
Sea $a$ un número real mayor que $1$ y 
$x \in \R$ arbitrario. Para el primer punto basta considerar dos números reales $s$ y $t$ tales que $s<x<t$ y por el Lema \ref{P:densidad de Q en R} existen racionales $r$ y $r'$ tales que 
$s<r<x<r'<t$, con lo cual es válido el punto. 

Para el segundo punto se requiere un poco más de astucia puesto que queremos controlar la cantidad $a^{r'} - a^{r}$, para lo cual el truco es tomar de forma ingeniosa los números $s$ y $t$ que mencionamos antes.

Tomemos en principio un número natural $n$ que definiremos luego con mayor precisión. La idea es no alejarnos tanto de $x$ con lo cual tomemos $s=x-\frac{1}{2n}$ y $t=x+\frac{1}{2n}$ y tenemos $r$ y $r'$ con las siguientes propiedades:
\[ x-\frac{1}{2n} <r<x<r'< x+\frac{1}{2n}\,;\]
\[ r'-r < x+\frac{1}{2n} - 
\Big( x-\frac{1}{2n} \Big) = \frac1n\,.\]
Sea $\epsilon >0$ dado. Dado que 
\[ a^{r'} - a^{r} = a^{r'-r+r} - a^{r}
= a^{r'-r}a^{r} - a^{r}
=(a^{r'-r}-1)\,a^{r}\,,\]
debemos estimar cada término de manera que el producto sea menor que $\epsilon$.
Sea $M$ cualquier número racional fijo mayor que $x$. Entonces $a^r < a^M$. Por otro lado, como $r'>r$ entonces $r'-r>0$ y $a^{r'-r}>1$. Así, usando la desigualdad de Bernoulli \ref{P:Bernoulli} con $h=a^{r'-r}-1>0$ se tiene
\[ a^{r'-r}-1 \leq \frac{(a^{r'-r})^n-1}{n} < \frac{a^{\frac1n n}-1}{n} = \frac{a-1}{n}\,. \]
Con todo esto, por la Proposición 
\ref{P:exponencial-monotona} y el Corolario \ref{1sobrenmenorqueeps}, es posible elegir $n$ suficientemente grande tal que 
\[  0<a^{r'} - a^{r} < \frac{a-1}{n}\,a^M < \epsilon\,,\]
y con esto queda probado el lema.
\end{proof}

Supongamos ahora que ya está definido $a^x$ para todo $x\in\R$, y $a>1$, de modo que valga que si $x<y$, entonces $a^x<a^y$.
Consideremos el conjunto $A=\{a^r: r\in\Q, r\le x\}$, formado por todas las potencias $a^r$ para $r$ racional y $r\le x$.
Por lo que acabamos de decir, resulta que $a^r\le a^x$, para todo $r\le x$ y luego $a^x$ es cota superior del conjunto $A$.
Si dado $\epsilon > 0$ elegimos $r$ y $r'$ como en el Lema~\ref{L:exponencial-continuidad}  resulta que 
\[
    0 < a^x - a^r \le a^{r'} - a^r < \epsilon ,
\]
y luego $a^x$ es el supremo del conjunto $A$.
Observamos entonces que si queremos que se cumpla la Proposición~\ref{P:exponencial-monotona} para todos los exponentes reales, debe ser necesariamente $a^x= \sup\{a^r: r\in\Q, r\le x\}$, si $a>1$, y análogamente $a^x= \inf\{a^r: r\in\Q, r\le x\}$ si $0<a<1$.

\begin{definition}
    Sea $a\in\R$, $a>0$ y $x \in \R$, entonces definimos
    \begin{itemize}
        \item $a^x= \sup\{a^r: r\in\Q, r\le x\}$, si $a>1$;
        \item $a^x= \inf\{a^r: r\in\Q, r\le x\}$, si $0<a<1$.
    \end{itemize}
\end{definition}

A partir de esta definición, que se basa en las anteriores, se puede probar ahora que valen todas las propiedades que enunciamos anteriormente, para $a\in\R$, $a>0$, y $x\in\R$, que resumimos a continuación:

\begin{theorem}\label{T:exponencial propiedades}
    Sea $a\in\R$, $a>0$, entonces:
    \begin{itemize}
        \item $a^{x+y} = a^x \cdot a^y$, si $x,y\in\R$;
        \item $a^{x\cdot y} = \big(a^x\big)^y$, si $x,y\in\R$;
        \item Si $a>1$ y $x<y$, entonces $a^x < a^y$;
        \item Si $0<a<1$ y $x<y$, entonces $a^x > a^y$;
        \item Si $a>1$, $a^x= \sup\{a^r: r\in\Q, r\le x\} = \inf\{a^r: r\in\Q, r\ge x\}$;
        \item Si $0<a<1$, $a^x= \sup\{a^r: r\in\Q, r\ge x\} = \inf\{a^r: r\in\Q, r\le x\}$;
    \end{itemize}
\end{theorem}

\begin{proof}
    Ejercicio.
\end{proof}

\subsubsection*{Ejercicios de la sección~\getcurrentref{chapter}.\getcurrentref{section}}

\begin{enumerate}
    \item* Sean $a$ y $b$ números reales positivos. Probar que para todo racional $t$ se cumple:
    \[
        \textbf{(a)}\ (ab)^t = a^tb^t.
        \qquad
        \textbf{(b)}\ a^{-t} = \big(a^{-1}\big)^t = \big(a^t\big)^{-1}.
        \qquad
        \textbf{(c)}\ (a/b)^t = a^t/b^t.
    \]

\end{enumerate}

\section{Valor absoluto y propiedades útiles}

\begin{definition}
    Dado un número real $x$, llamaremos \emph{módulo} o \emph{valor absoluto} de $x$ al mismo $a$ si $a$ es no negativo o al opuesto $-a$ si $a$ es negativo
    \[
        |x| = \begin{cases}
            x, \quad&\text{si }x\ge 0,\\
            -x, \quad&\text{si }x< 0.
        \end{cases}
    \]
\end{definition}

Observemos que cualquiera sea $x$, resulta $|x|\ge 0$, y $0$ es el único número cuyo módulo es cero. En otras palabras, $|x|>0$, para todo $x\neq 0$, y $|0|=0$.

Observamos que siempre se cumple $x \le |x|$, puesto que si $x < 0$, entonces, como 
$0\le |x|$, por la transitividad resulta $x < |x|$; y si $x \ge 0$, resulta que $x = |x|$.


Si pensamos en la representación de $\R$ como una recta, el módulo de $x$ es la distancia que hay entre $x$ y $0$, y más aún $|x-y|$ es la distancia que hay entre $x$ e $y$.

También es fácil ver que $|a|=\sqrt{a^2}$.
Utilizando esta igualdad es fácil demostrar que 
\begin{itemize}
    \item $|x\,y|=|x|\,|y|$, cualesquiera sean $x,y\in\R$.
    \item $|x/y|=|x|/|y|$, cualesquiera sean $x,y\in\R$, $y\neq 0$.
\end{itemize}

Una desigualdad muy sencilla y también muy útil es la \emph{desigualdad triangular}, que se enuncia en la siguiente proposición.

\begin{proposition}
    Si $x,y\in \R$, entonces 
    \[ |x+y| \le |x|+|y|.
    \]
\end{proposition}

\begin{proof}
    Supongamos por el contrario que existen $x,y\in \R$ tales que $|x+y|>|x|+|y|$.
    Claramente, si esto ocurre, $x\neq 0$ y $y\neq 0$, por lo que ambos términos de la desigualdad son positivos, 
    y por lo tanto $|x+y|^2>(|x|+|y|)^2$. O sea
    \[
        |x+y|^2 > |x|^2 + 2\,|x|\,|y| + |y|^2 = |x|^2 + 2\,|x\,y| + |y|^2 .
    \]
    Usando ahora que $|a|=\sqrt{a^2}$ para cualquier $a\in \R$, resulta
    \[
        \big(\sqrt{(x+y)^2}\big)^2 > \big(\sqrt{x^2}\big)^2 + 2\, |x\,y| + \big(\sqrt{y^2}\big)^2.
    \]
    Y ahora podemos usar que para cualquier número $a>0$, resulta $\big(\sqrt{a})^2 = a$, que implica
    \[
        \big(x+y\big)^2 > x^2 + 2\, |x\,y| + y^2.
    \]
    Es decir,
    \[
        x^2+ 2\,x\,y + y^2 > x^2 + 2\, |x\,y| + y^2,
    \]
    que a su vez implica $x\,y > |x\,y|$. Esto contradice el hecho que $a \le |a|$ para todo $a\in\R$.
    Esta contradicción provino de suponer que $|x+y|>|x|+|y|$, por lo que queda demostrada la afirmación de esta proposición.
\end{proof}

\subsubsection*{Ejercicios de la sección~\getcurrentref{chapter}.\getcurrentref{section}}

\begin{enumerate}
    \item* Demostrar que si $b$ es positivo, entonces
    \[
        |a| \le b \quad\iff\quad -b \le a \le b.
    \]

    \item\label{ej:triangular resta} Demostrar que $\big| |a|-|b| \big| \le |a-b|$, cualesquiera sean los números $a,b\in\R$.



    \item Verificar que:
    \begin{enumerate}
        \item $\Big|\frac1n - 0\Big| < \frac1{10}$, para todo $n\in\N$, $n > 10$.
        \item $\Big|\frac1{n^2} - 0\Big| < \frac1{10}$, para todo $n\in\N$, $n > 10$.
        \item $\Big|\frac{n + 5}n-1 \Big|< 1$, para todo $n\in\N$, $n > 5$.
        \item $\Big|\frac{n - 10}n-1\Big| < \frac12$, para todo $n\in\N$, $n > 20$.
        \item $\Big|\frac{n+n^2}{n^2} - 1\Big| < \frac1{100}$, para todo $n\in\N$, $n > 100$.
        \item $\Big| \frac{5n^2 + n - 5}{n^2}-5\Big| < 1 $, para todo $n\in\N$, $n > 5$.
        \item $\Big| \frac{4n^3+2n^2-6}{3n^3+10} - \frac43\Big|  < \frac1{5}$, para todo $n\in\N$, $n > 5$.
    \end{enumerate}


    \item Hallar $N_0\in\N$ tal que
        \begin{enumerate}
            \item $\Big|\frac1n - 0\Big| < \frac1{100}$, para todo $n\in\N$, $n > N_0$.
            \item $\Big|\frac1{n^2} - 0\Big| < \frac1{100}$, para todo $n\in\N$, $n > N_0$.
            \item $\Big|\frac{n + 5}n-1 \Big|< \frac1{10}$, para todo $n\in\N$, $n > N_0$.
            \item $\Big|\frac{n - 10}n-1\Big| < \frac1{100}$, para todo $n\in\N$, $n > N_0$.
            \item $\Big|\frac{n+n^2}{n^2} - 1\Big| < \frac1{1000}$, para todo $n\in\N$, $n > N_0$.
            \item $\Big| \frac{5n^2 + n - 5}{n^2}-5\Big| < \frac1{10} $, para todo $n\in\N$, $n > N_0$.
            \item $\Big| \frac{4n^3+2n^2-6}{3n^3+10} - \frac43\Big|  < \frac1{100}$, para todo $n\in\N$, $n > N_0$.
        \end{enumerate}

    \item Dado $\alpha \in \R$, $\alpha > 0$, hallar $N_0\in\N$ tal que
    \begin{enumerate}
            \item $\Big|\frac1n - 0\Big| < \alpha$, para todo $n\in\N$, $n > N_0$.
            \item $\Big|\frac1{n^2} - 0\Big| < \alpha$, para todo $n\in\N$, $n > N_0$.
            \item $\Big|\frac{n + 5}n-1 \Big|< \alpha$, para todo $n\in\N$, $n > N_0$.
            \item $\Big|\frac{n - 10}n-1\Big| < \alpha$, para todo $n\in\N$, $n > N_0$.
            \item $\Big|\frac{n+n^2}{n^2} - 1\Big| < \alpha$, para todo $n\in\N$, $n > N_0$.
            \item $\Big| \frac{5n^2 + n - 5}{n^2}-5\Big| < \alpha$, para todo $n\in\N$, $n > N_0$.
            \item $\Big| \frac{4n^3+2n^2-6}{3n^3+10} - \frac43\Big|  < \alpha$, para todo $n\in\N$, $n > N_0$.
    \end{enumerate}

\end{enumerate}


% \mara{Se me ocurre que se podría crear una subsección que sea `Ejercicios' que tenga los de la sección 1.5, y nuclear todos los ejercicios del capítulo en una subsección `Ejercicios del capítulo' o algo así.}
% Buena idea. Ya lo hice en este capítulo.

% \mauri{Algo que después usan consecuencia de la desigualdad triangular es que $|a+b| \geq |a|-|b|$ que sale redefiniendo simplemente $x=a+b$ e 
% $y=-b$ y reordenando la desigualdad. Tal vez se puede poner como observación o como ejercicio.} 
% está como ejercicio en la lista de ejercicios.

\section{Ejercicios del capítulo~\getcurrentref{chapter}}

\begin{enumerate}

    \item Probar que si $a,b,x,y \in \R$, $a<b$ y $x<y$, entonces $a+x < b+y$.

    \item Probar que si $a,b\in\R$ y $a<b$, entonces $\D a<\frac{a+b}2<b$.

    \item Verificar que:
    \begin{enumerate}
        \item $n < n + 5 < 2 n$, para todo $n\in\N$, $n > 5$.
        \item $n < n + 5 < 2 n$, para todo $n\in\N$, $n > 50$.
        \item $\frac n2 < n - 10 < n$, para todo $n\in\N$, $n > 20$.
        \item $\frac n2 < n - 10 < n$, para todo $n\in\N$, $n > 100$.
        \item $5\, n^2 < 5\,n^2 + n - 5 < 6\, n^2$, para todo $n\in\N$, $n > 5$.
        \item $4\, n^2 < 5\,n^2 - n  < 5\, n^2$, para todo $n\in\N$, $n > 5$.
    \end{enumerate}

    \item Hallar $N_0\in\N$ tal que:
    \begin{enumerate}
        \item $n < n + 30 < 2 n$, para todo $n\in\N$, $n > N_0$.
        \item $\frac n2 < n - 30 < n$, para todo $n\in\N$, $n > N_0$.
        \item $5\, n^2 < 5\,n^2 + n - 5 < 6.1\, n^2$, para todo $n\in\N$, $n > N_0$.
        \item $4.9\, n^2 < 5\, n^2 - n  < 5\, n^2$, para todo $n\in\N$, $n > N_0$.
        \item $n^2 < 2\,n^2 - 10\, n + 8 < 3\, n^2$, para todo $n\in\N$, $n > N_0$.
        \item $2\, n^3 < 3\,n^3 - 5 n^2 + 30 < 3\, n^3$, para todo $n\in\N$, $n > N_0$.
    \end{enumerate}



    \item Probar que el intervalo $(0,1)$ es acotado inferiormente pero no tiene mínimo.

    \item Probar que el intervalo $(0,1)$ es acotado superiormente pero no tiene máximo.

    \item Sea $A$ un conjunto de números reales con mínimo $m$. Sea $A'$ el conjunto definido por $A'=\{-x : x \in A\}$. Probar que $\max A' = -\min A$.


    \item Sea $A$ un subconjunto acotado de $\R$ y sea $B\subseteq A$ no vacío. Probar que: 
    \begin{enumerate}
        \item Toda cota superior de $A$ es cota superior de $B$.
        \item Toda cota inferior de $A$ es cota inferior de $B$.
    \end{enumerate}
    
    \item \mara{Sea $A$ un subconjunto acotado de $\R$ y sea $B\subseteq A$ no vacío. Probar que $\inf{A}\leq\inf{B}\leq\sup{B}\leq\sup{A}$.}

    \item* Enunciar y demostrar una proposición análoga a la Proposición~\ref{P:supremo-caracterizacion} para el ínfimo.

    \item \mara{Sea $a$ un número real tal que $0\leq a <\epsilon$ para todo $\epsilon>0$. 
        Probar que $a=0$. 
            
    Análogamente, demostrar que si $a$ y $b$ son dos números reales tales que $a<b+\epsilon$ para todo $\epsilon>0$, entonces $a\leq b$.}
    \item* Sean $a$ y $b$ números reales positivos. Probar que para todo racional $t$ se cumple:
    \[
        \textbf{(a)}\ (ab)^t = a^tb^t.
        \qquad
        \textbf{(b)}\ a^{-t} = \big(a^{-1}\big)^t = \big(a^t\big)^{-1}.
        \qquad
        \textbf{(c)}\ (a/b)^t = a^t/b^t.
    \]

    \item* Demostrar que si $b$ es positivo, entonces
    \[
        |a| \le b \quad\iff\quad -b \le a \le b.
    \]

    \item\label{ej:triangular resta} Demostrar que $\big| |a|-|b| \big| \le |a-b|$, cualesquiera sean los números $a,b\in\R$.



    \item Verificar que:
    \begin{enumerate}
        \item $\Big|\frac1n - 0\Big| < \frac1{10}$, para todo $n\in\N$, $n > 10$.
        \item $\Big|\frac1{n^2} - 0\Big| < \frac1{10}$, para todo $n\in\N$, $n > 10$.
        \item $\Big|\frac{n + 5}n-1 \Big|< 1$, para todo $n\in\N$, $n > 5$.
        \item $\Big|\frac{n - 10}n-1\Big| < \frac12$, para todo $n\in\N$, $n > 20$.
        \item $\Big|\frac{n+n^2}{n^2} - 1\Big| < \frac1{100}$, para todo $n\in\N$, $n > 100$.
        \item $\Big| \frac{5n^2 + n - 5}{n^2}-5\Big| < 1 $, para todo $n\in\N$, $n > 5$.
        \item $\Big| \frac{4n^3+2n^2-6}{3n^3+10} - \frac43\Big|  < \frac1{5}$, para todo $n\in\N$, $n > 5$.
    \end{enumerate}


    \item Hallar $N_0\in\N$ tal que
        \begin{enumerate}
            \item $\Big|\frac1n - 0\Big| < \frac1{100}$, para todo $n\in\N$, $n > N_0$.
            \item $\Big|\frac1{n^2} - 0\Big| < \frac1{100}$, para todo $n\in\N$, $n > N_0$.
            \item $\Big|\frac{n + 5}n-1 \Big|< \frac1{10}$, para todo $n\in\N$, $n > N_0$.
            \item $\Big|\frac{n - 10}n-1\Big| < \frac1{100}$, para todo $n\in\N$, $n > N_0$.
            \item $\Big|\frac{n+n^2}{n^2} - 1\Big| < \frac1{1000}$, para todo $n\in\N$, $n > N_0$.
            \item $\Big| \frac{5n^2 + n - 5}{n^2}-5\Big| < \frac1{10} $, para todo $n\in\N$, $n > N_0$.
            \item $\Big| \frac{4n^3+2n^2-6}{3n^3+10} - \frac43\Big|  < \frac1{100}$, para todo $n\in\N$, $n > N_0$.
        \end{enumerate}

    \item Dado $\alpha \in \R$, $\alpha > 0$, hallar $N_0\in\N$ tal que
    \begin{enumerate}
            \item $\Big|\frac1n - 0\Big| < \alpha$, para todo $n\in\N$, $n > N_0$.
            \item $\Big|\frac1{n^2} - 0\Big| < \alpha$, para todo $n\in\N$, $n > N_0$.
            \item $\Big|\frac{n + 5}n-1 \Big|< \alpha$, para todo $n\in\N$, $n > N_0$.
            \item $\Big|\frac{n - 10}n-1\Big| < \alpha$, para todo $n\in\N$, $n > N_0$.
            \item $\Big|\frac{n+n^2}{n^2} - 1\Big| < \alpha$, para todo $n\in\N$, $n > N_0$.
            \item $\Big| \frac{5n^2 + n - 5}{n^2}-5\Big| < \alpha$, para todo $n\in\N$, $n > N_0$.
            \item $\Big| \frac{4n^3+2n^2-6}{3n^3+10} - \frac43\Big|  < \alpha$, para todo $n\in\N$, $n > N_0$.
    \end{enumerate}

\end{enumerate}



