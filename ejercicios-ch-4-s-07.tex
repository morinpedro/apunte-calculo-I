\item Determinar cuáles de las siguientes funciones cumplen que $f(h)=o(h)$:
\begin{multicols}{2}
\begin{enumerate}
    % \item $f(x) = x^2$
    \item $f(x) = x^3$
    \item $f(x) = 1$
    \item $f(x) = |x|^{1.001}$
    \item $f(x) = |x|^\alpha$ con $\alpha > 1$
    \item $f(x) = |x|^{1/3}$
    \item $f(x) = |x|$
    \item $f(x) = |x|^\alpha$ con $\alpha \le 1$
    \item $f(x) = \frac{x}{\log |x|}$
    % \item $f(x) = 1-\cos(x)$
    \item $f(x) = \log(|x|)$
    \item $f(x) = e^x$
    \item $f(x) = e^{-1/|x|}$
    
    % \item $f(x) = \sen(x)$, 
\end{enumerate}    
\end{multicols}

\item Demostrar las siguientes afirmaciones:
    \begin{enumerate}
        \item Si $f(h)=o(h)$, entonces $cf(h) = o(h)$.
        \item Si $f(h)=o(h)$ y $g(h) = o(h)$, entonces $f(h)+g(h) = o(h)$.
        \item Si $f(h)=o(h)$ y $g(h) = o(h)$, entonces $f(h)g(h) = o(h)$.
    \end{enumerate}
